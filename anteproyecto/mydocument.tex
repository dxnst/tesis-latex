\documentclass[11pt,letterpaper]{report}
\usepackage[margin=1in]{geometry}
\usepackage[utf8]{inputenc}
\usepackage[T1]{fontenc}
\usepackage{helvet}
\renewcommand{\familydefault}{\sfdefault}
\linespread{1.5}
\usepackage[spanish,es-nodecimaldot]{babel}
\usepackage{titlesec}
\usepackage[hidelinks]{hyperref}
\usepackage[skip=0pt plus8pt, indent=1.25cm]{parskip}


% Formatos de títulos
\titleformat{\chapter}[block]{\normalfont\fontsize{12pt}{12pt}\bfseries}{SECCIÓN \ \thechapter.}{0.6em}{\centering\MakeUppercase}{
\titlespacing*{\chapter}{0em}{*3}{*2}
\titleformat{\section}[block]{\normalfont\fontsize{12pt}{12pt}\bfseries}{\thesection}{0.6em}{}
\titlespacing*{\section}{0em}{*2}{*1.2}
\titleformat{\subsection}[block]{\normalfont\bfseries\itshape}{\thesubsection}{0.6em}{}
\titlespacing*{\subsection}{4em}{*1}{*0.6}

% Formato de citación
\usepackage{citation-style-language}
\cslsetup{style = mary-ann-liebert-vancouver}
\addbibresource{mydocument.bib}

% Formato de numeración
\renewcommand*{\theenumi}{\thesection.\arabic{enumi}}
\renewcommand*{\theenumii}{\theenumi.\arabic{enumii}}

% Cosas generales
\title{Factores de riesgo en el neurodesarrollo infantil}
\author{Soto Consuegra, Josué Daniel \and López Castillo, Sarah Ivón \and
Ixquiac Vásquez, Etelvina Del Rosario \and Guzmán Pérez, Mariana Del Rosario
\and Mazariegos Manrique, Sonia María}
\newcommand{\tiempito}{marzo a junio de 2025}
\newcommand{\muestradeseada}{1,697}
\newcommand{\asq}{“Cuestionario Edades y Etapas 3”}

% Formato de expresiones matemáticas
\usepackage{amsmath}
\usepackage{graphicx} % \scalebox
\usepackage{environ}
\NewEnviron{myequation}{%
\begin{equation}
\scalebox{1.5}{$\BODY$}
\end{equation}
}

\begin{document}
	\tableofcontents
	\maketitle
	\chapter{Antecedentes y contextualización del problema de investigación}
En un estudio transversal realizado por Mehner et al. (2019), titulado "La
asociación de la puntuación de riesgo acumulativo con los resultados de ASQ-3
en una región rural empobrecida de Guatemala", se evaluó una muestra de
conveniencia de 148 madres con niños de 12 a 52 meses de edad en una zona
rural de Guatemala. El objetivo principal fue desarrollar una puntuación de
riesgo compuesta por factores fácilmente obtenibles para diseñar intervenciones
e identificar a los niños de alto riesgo que más se beneficiarían de estas.
Se utilizaron encuestas de interacción madre-hijo y \asq\ para evaluar el
desarrollo. Los resultados mostraron que el 58\% de los niños tenían
puntuaciones anormales en $\ge$1 dominio del ASQ-3, y el 35\% en $\ge$2
dominios. Se desarrollaron tres puntuaciones de riesgo: Riesgo Demográfico
Materno (DR), Interacción Madre-Hijo (MCI) y Riesgo Combinado (CR). La
probabilidad de tener $\ge$2 dominios con puntuaciones anormales aumentó
significativamente con un puntaje DR creciente (OR, 1.46 [IC 95\%, 1.15-1.86]
p<0.05) y un puntaje CR creciente (OR, 2.08 [IC 95\%, 1.41-3.07], p<0.05). Los
autores concluyeron que un índice de riesgo acumulativo combinado de factores
demográficos e interacciones madre-hijo parece ser una herramienta útil para
predecir qué niños tienen puntuaciones anormales en múltiples dominios del
desarrollo. \cite{CMehner2019}

En una revisión sistemática y meta-análisis realizada por Wondmagegn et al.
(2024), titulada "Prevalencia y determinantes del retraso del desarrollo entre
los niños en países de ingresos bajos y medios: una revisión sistemática y un
metanálisis", se analizaron 21 estudios primarios publicados entre 2010 y
2024, involucrando a un total de 54,067 niños en países de ingresos bajos y
medios. El objetivo principal fue evaluar la prevalencia combinada del retraso
del desarrollo confirmado y sus determinantes entre los niños en estos países.
Los resultados mostraron una prevalencia combinada de retraso del desarrollo
del 18.83\% (IC 95\%: 15.53-22.12\%). En el análisis de subgrupos, se observó
una alta prevalencia de retraso del desarrollo [26.69\% (IC 95\%: 15.78-37.60)]
en estudios realizados en África. Los determinantes significativos del retraso
del desarrollo fueron la educación materna [OR: 3.04; IC 95\% (2.05, 4.52)] y
el bajo peso al nacer [OR: 3.61; IC 95\% (1.72, 7.57)]. Los autores concluyeron
que la prevalencia combinada de retraso del desarrollo en países de ingresos
bajos y medios era alta en comparación con los países de altos ingresos,
especialmente en África, y que el nivel educativo materno y el peso al nacer
estaban significativamente asociados con los retrasos del desarrollo.
\cite{Wondmagegn2024}

En un estudio transversal comunitario realizado en áreas urbanas de Etiopía por
Delbiso et al. (2024), titulado "Desarrollo de la primera infancia y estado
nutricional en la Etiopía urbana", se evaluaron 627 pares de madres e hijos de
12-36 meses de edad entre julio y septiembre de 2022. El desarrollo infantil
temprano (DIT) se evaluó utilizando el \asq, mientras que el estado nutricional
se determinó mediante mediciones antropométricas. Los resultados mostraron que
los retrasos en los dominios del DIT eran comunes, especialmente en el dominio
motor fino (41.9\%). Más de la mitad de los niños (52.8\%) presentaban retraso
en el crecimiento. Se encontró que el retraso en el crecimiento y el bajo peso
estaban asociados con retrasos en el DIT, mientras que la desnutrición aguda no
lo estaba. Los niños con retraso en el crecimiento tenían más probabilidades de
tener peores retrasos en el DIT en los dominios motor fino (OR = 1.54; IC 95\%:
1.11-2.15), motor grueso (OR = 1.47;IC 95\%: 1.05-2.04) y resolución de
problemas (OR = 1.41; IC 95\%: 1.02-1.96) en comparación con los niños sin
retraso en el crecimiento. De manera similar, los niños con bajo peso tenían
más probabilidades de tener peores retrasos en el DIT en los dominios motor
grueso (OR = 1.91; IC 95\%: 1.20-3.04) y motor fino (OR = 1.90; IC 95\%:
1.15-3.15) en comparación con los niños de peso normal. \cite{Delbiso2024}

Domek et al. (2023) realizaron un estudio piloto para evaluar los efectos a
largo plazo de una intervención simple con títeres de dedo para promover el
desarrollo infantil temprano en el ámbito de atención primaria. La muestra
incluyó 172 niños de familias principalmente de bajos ingresos, divididos en
cohortes de intervención temprana (2 meses) y tardía (6 o 12 meses). Se utilizó
el \asq\ para evaluar el desarrollo infantil hasta los 36 meses. Los resultados
mostraron que la intervención temprana se asoció con mejores trayectorias de
desarrollo socioemocional en comparación con
la intervención tardía (diferencia en pendiente de 0.12, p=0.018). También se
observaron diferencias que se acercaron a la significancia estadística en
comunicación (p=0.056) y en la puntuación combinada no motora (p=0.052). No se
encontraron diferencias significativas en los dominios de resolución de
problemas, motricidad gruesa y fina. Los autores concluyeron que la
intervención con títeres de dedo puede proporcionar una forma simple, de bajo
costo y escalable de fomentar interacciones cuidador-infante que promuevan el
desarrollo del lenguaje y socioemocional, especialmente cuando se proporciona
en la infancia temprana. Este estudio destaca la importancia de las
intervenciones tempranas en atención primaria y su potencial impacto en el
desarrollo infantil a largo plazo. \cite{Domek2023}

En un estudio transversal, descriptivo y exploratorio realizado por Ramos y
Della Barba (2021), titulado Cuestionarios de edades y etapas de Brasil en el
seguimiento del desarrollo en la primera infancia, se analizaron 392 niños de 
5 a 50 meses de edad que asistían a 6 Centros de Educación Infantil (CEIs) en
un municipio del interior del estado de São Paulo, Brasil. El objetivo
principal fue delinear el perfil del desarrollo global de los niños utilizando 
el \asq\ edición Brasil (ASQ-BR) y verificar la aplicabilidad
de este instrumento por parte de los maestros preescolares. Los resultados
mostraron que la mayoría de los niños presentaron un desarrollo dentro de lo 
esperado, con los mejores desempeños en los dominios de Motricidad Gruesa
(79.44\%), Comunicación (72.34\%) y Resolución de Problemas (69.54\%). Sin
embargo, se observó una incidencia significativa de riesgo en los dominios
Personal-Social (22.08\%) y Motricidad Fina (19.03\%). En el análisis por sexo,
las niñas obtuvieron puntuaciones significativamente más altas que los niños en
los dominios de Motricidad Fina y Personal-Social. Los autores concluyeron que
el ASQ-BR se presenta como un instrumento potencial para el cribado del
desarrollo infantil en guarderías y preescolares, permitiendo a los
profesionales reflexionar sobre su propia práctica y atender mejor las
necesidades individuales de los niños. \cite{RAMOS2021}

Oumer et al. (2022), titulado “El retardo de crecimiento y bajo peso, pero no
la desnutrición aguda, están asociados con el retraso en el desarrollo
infantil en el suroeste de Etiopía”, se analizaron 507 pares de madres e hijos
en el Suroeste de Etiopía. El objetivo principal fue identificar la relación
entre diferentes formas de malnutrición y el retraso en el desarrollo infantil
entre niños de 12 a 59 meses de edad. Los resultados mostraron una prevalencia
de retraso en el desarrollo del 29.4\% (IC 95\%: 25.4-33.4\%). En el análisis
de subgrupos, se observaron retrasos en el desarrollo de habilidades motoras
gruesas (17.2\%), comunicación (16.8\%), resolución de problemas (13.4\%),
habilidades personales-sociales (10.8\%) y motricidad fina (10.1\%).
Los determinantes significativos del retraso en el desarrollo fueron el trabajo
materno fuera del hogar [AOR: 2.9; IC 95\% (1.8, 4.8)], el nacimiento prematuro
[AOR: 3.2; IC 95\% (1.4, 7.0)], la iniciación temprana de la alimentación
complementaria [AOR: 2.5; IC 95\% (1.37, 4.6)], el retraso en el crecimiento
[AOR: 3.0; IC 95\% (1.9, 4.7)], el bajo peso [AOR: 2.3; IC 95\% (1.1, 4.7)] y
una baja puntuación de diversidad dietética [AOR: 3.1; IC 95\% (1.3, 7.5)].
Los autores concluyeron que el retraso en el desarrollo infantil es un problema
de salud pública en la región y está fuertemente asociado con la desnutrición
crónica, el bajo peso, el consumo de una dieta poco diversificada y prácticas
subóptimas de alimentación infantil.
\cite{Oumer2022}

En el distrito de Quetzaltenango, Guatemala, la vulnerabilidad económica, el
acceso limitado a servicios de salud y los factores nutricionales representan
un riesgo significativo para el neurodesarrollo infantil. Al igual que en otros
países de ingresos bajos y medios, las condiciones adversas en esta región
pueden influir negativamente en los hitos del desarrollo infantil temprano. En
este contexto, se plantea la necesidad de estudiar y caracterizar los factores
de riesgo en el neurodesarrollo infantil en niños menores de 5 años que acuden
a servicios de salud pública en esta región, contribuyendo a la evidencia que
respalde intervenciones efectivas en salud pública para mitigar estos riesgos.

Por lo tanto, se formula la pregunta de investigación: ¿Cuáles son los factores
de riesgo asociados al neurodesarrollo en niños menores de 5 años que asisten a
servicios de salud pública en el distrito de Quetzaltenango durante el periodo
de \tiempito?

	\chapter*{Justificación}
Los trastornos del desarrollo, también conocidos como retrasos del desarrollo,
constituyen un grupo heterogéneo de condiciones que afectan el aprendizaje, el
lenguaje, el comportamiento o las habilidades motoras.
\cite{cdcDevelopmentalDisability} Estos retrasos se identifican cuando un niño
no alcanza los hitos de desarrollo esperados en comparación con sus pares de la
misma población \cite{DevelopmentalSurveillance}. Por ello es importante
destacar que el retraso en el desarrollo no es un diagnóstico en sí mismo, sino
un término descriptivo utilizado en la práctica clínica para indicar un
fenotipo amplio que requiere una evaluación más detallada para determinar las
áreas específicas de desarrollo afectadas. Hay tres tipos de retraso en el
desarrollo basado en el número de dominios involucrados: 1) Retraso aislado en
el desarrollo: involucra un solo dominio; 2) Múltiples retrasos en el
desarrollo: 2 o más dominios o líneas de desarrollo afectados; y, 3) Retraso
global en el desarrollo: retraso significativo en la mayoría de los dominios de
desarrollo. \cite{Bellman2013} Aunque la etiología de la mayoría de los
retrasos en el desarrollo es idiopática, cuando se identifica, puede incluir
factores genéticos, ambientales y/o psicosociales. \cite{DevelopmentalDelay}

En Guatemala, según el informe de la línea de base de la Gran Cruzada Nacional
por la Nutrición 2021/2022 de la Secretaría de Seguridad Alimentaria y
Nutricional, solo el 1.9\% de las madres de niños entre 2 y 5 años reportaron
que sus hijos habían asistido alguna vez a un programa de primera infancia, y
apenas el 0.6\% asiste actualmente a un Centro Comunitario de Desarrollo
Infantil Temprano. Más preocupante aún, solo el 49.8\% de los niños de 24 a 59
meses se encuentran en el camino adecuado de desarrollo, salud, aprendizaje y
bienestar psicosocial. \cite{SESAN2022}

A nivel global, según un reporte de UNICEF en 2023, se estima que 250 millones
de niños menores de 5 años están en riesgo de no alcanzar su potencial de
desarrollo. Aproximadamente 200 millones de niños menores de 5 años no están
creciendo, no presentan un adecuado desarrollo global, debido a la desnutrición
en la primera infancia. Además, más de 2 de cada 5 niños entre 3 y 4 años no
reciben la estimulación temprana ni el cuidado parental adecuados. Como
resultado de estas y otras amenazas, el 29\% de los niños de 3 a 5 años no
están logrando un desarrollo apropiado. \cite{UNICEF2023}

El neurodesarrollo infantil es un proceso complejo y dinámico que sienta las
bases para el futuro cognitivo, emocional y social de los individuos. En
Quetzaltenango, Guatemala, existe una brecha significativa en la investigación
sobre los factores que influyen en el desarrollo neurológico de los niños
menores de 5 años. Esta carencia de datos locales específicos obstaculizan la
implementación de intervenciones efectivas y políticas públicas adecuadas.

Para llevar a cabo este estudio en Quetzaltenango, es necesario un equipo de 5
investigadores debido a la complejidad y el alcance de la muestra, la cual
comprende \muestradeseada\ niños. La distribución del trabajo se detalla a
continuación:

\begin{itemize}
    \item Carga de trabajo y distribución

Cada investigador estará a cargo de evaluar aproximadamente 339 niños, lo cual
permite una división equitativa para asegurar una atención detallada en cada
caso. Esto es crucial para mantener la calidad de los datos y la consistencia
en la recolección de información, aspecto necesario para la validez del
estudio.

	\item Tiempo estimado de evaluación

Cada evaluación individual tomará alrededor de 50 minutos, incluyendo el
consentimiento informado, perfil social, aplicación del \asq, y las actividades
específicas. Esto representa aproximadamente 1,417 horas en total o 283 horas
por investigador, distribuidas en 14 semanas con jornadas de evaluación diaria.
La presencia de 5 investigadores optimiza el proceso y asegura que las
evaluaciones se realicen en el tiempo programado.
	
	\item Cobertura de múltiples puntos de atención

La investigación se llevará a cabo en tres servicios de salud de Quetzaltenango:
el Centro de Salud de Quetzaltenango, el Puesto de Salud de Pacajá y el Puesto
de Salud de San José Chiquilajá. La asignación de varios investigadores a estos
puntos asegura que las evaluaciones sean eficientes y que se alcance una
cobertura geográfica completa en un tiempo limitado.

	\item Atención a casos en riesgo

Los investigadores deben proporcionar plan educacional y material de
seguimiento para niños identificados con riesgo en el desarrollo. La promoción
de la Guía de Estimulación Oportuna de UNICEF, será una labor compartida entre
los investigadores y garantizará la entrega de recomendaciones adecuadas a los
padres de los niños participantes.

\end{itemize}

En conclusión, la integración de un equipo de 5 investigadores permite abordar
de manera exhaustiva y precisa los desafíos de la evaluación de neurodesarrollo
en Quetzaltenango. Los resultados esperados no solo aportarán evidencia
científica local, sino que también promoverán intervenciones que puedan mejorar
el desarrollo integral de los niños, sensibilizando a las autoridades y
profesionales de la salud sobre la importancia de una intervención temprana y
costo efectiva en la primera infancia.

\chapter{Objetivos}
\section{Objetivo general}
\begin{enumerate}
	\item Establecer la asociación entre factores sociodemográficos,
	económicos, familiares y médicos con el riesgo en el neurodesarrollo en
	niños menores de 5 años que asisten a servicios de atención primaria en el
	distrito de Quetzaltenango, mediante evaluaciones con el \asq\ durante
	2025.
\end{enumerate}
\section{Objetivos específicos}
\begin{enumerate}
	\item Clasificar los resultados del \asq\ según grupos de edad para
	detectar patrones específicos de riesgo en los dominios del
	neurodesarrollo. 
	\item Evaluar la asociación entre factores socioeconómicos, demográficos,
	ambientales y antecedentes perinatales y el riesgo de retraso en el
	neurodesarrollo utilizando el \asq.
	\item Analizar la relación entre acceso a servicios de atención primaria
	durante el periodo prenatal y postnatal con la presencia de retrasos en el
	neurodesarrollo.
\end{enumerate}

\chapter{Población y métodos}
\section{Enfoque y diseño de investigación}
La investigación tendrá un enfoque cuantitativo, diseño analítico,
observacional, de cohorte prospectivo.

\section{Población y muestra}

\begin{enumerate}
	\item Población o universo:
	1,701 niños menores de 5 años que acudan a servicios de atención primaria
	del distrito de salud de Quetzaltenango durante 2025.
%	\item Muestra:
%		Se realizará un muestreo no probabilístico por conveniencia, reclutando
%		a todos los niños que cumplan con los criterios de inclusión y asistan
%		a los servicios de atención primaria participantes durante \tiempito,
%		hasta alcanzar el tamaño de muestra deseado de \muestradeseada\ niños.
\end{enumerate}

%\section{Tipo y técnica de muestreo}
%Se utilizará un muestreo aleatorio simple probabilístico. El tamaño de la
%muestra se ha calculado aplicando la fórmula para poblaciones finitas, con base
%en una población total de 21,286 niños menores de 5 años según datos del INE
%para 2025 en el municipio de Quetzaltenango. \cite{INE} Se estableció un nivel
%de confianza del 97\%, una prevalencia estimada del 50\% y un margen de error
%del 3\%. La muestra se calculó a partir de la fórmula:

%% Ecuación para muestreo de poblaciones finitas
%\begin{myequation}%
%	n = \frac{N \times Z^2 \times p \times q}{e^2 \times (N-1) + Z^2 \times p \times q} = \frac{21{,}286 \times 2.58^2 \times 0.5 \times 0.5}{0.03^2 \times (21{,}286-1) + 2.58^2 \times 0.5 \times 0.5}
 %
%\end{myequation}
%
%Donde: \\
%n = tamaño de la muestra \\
%N = tamaño de la población (21,286) \\
%Z = valor crítico del nivel de confianza (99\% = 2.58) \\
%p = proporción esperada (50\% = 0.5) \\
%q = 1-p (proporción complementaria) (50\% = 0.5) \\
%e = margen de error (3\% = 0.03) \\

%% Ecuación para muestreo de poblaciones finitas
%\begin{myequation}%
%	n = \frac{35{,}422.0326}{19.1565 + 1.6641} = 1{,}701 %
%\end{myequation}
%

%El tipo de muestreo será no probabilístico por conveniencia, reclutando a todos
%los niños que cumplan con los criterios de inclusión y asistan a los servicios
%de atención primaria participantes durante \tiempito, hasta alcanzar el tamaño
%de muestra deseado de \muestradeseada\ niños.

\section{Criterios de inclusión y exclusión}
\begin{enumerate}
	\item \textbf{Criterios de inclusión:}
		\begin{itemize}
		\item Niños de 0 a 59 meses de edad que acuden a servicios de atención
		primaria para controles de crecimiento y desarrollo, vacunación o
		consulta médica.
		\item Padres o cuidadores que acepten participar en el estudio y firmen
		el consentimiento informado. 
		\end{itemize}
	\item \textbf{Criterios de exclusión:}
		\begin{itemize}
		\item Niños con diagnóstico previo de trastornos del neurodesarrollo o
		discapacidad intelectual. 
		\item Padres o cuidadores que no acepten participar en el estudio o se
		retiren durante el proceso.
		\end{itemize}
\end{enumerate}

\section{Variables}
Las variables a estudiar se han seleccionado para analizar su posible
influencia en el neurodesarrollo de los niños menores de 5 años en
Quetzaltenango, conforme a los objetivos planteados:
\begin{enumerate}
	\item Evaluación del neurodesarrollo será evaluado mediante el \asq\ que
	proporciona una evaluación sistemática del desarrollo infantil temprano en
	cinco áreas:
		\begin{enumerate}
			\item Desarrollo de la comunicación: habilidades lingüísticas
			receptivas y expresivas, incluyendo balbuceo, vocalización,
			escucha y comprensión, según la edad del niño.
			\item Desarrollo motor grueso: control postural, movimientos
			corporales amplios, equilibrio y coordinación general apropiados
			para cada etapa.
			\item Desarrollo motor fino: destreza manual, coordinación
			visomotora, manipulación de objetos y precisión de movimientos
			pequeños acorde a la edad.
			\item Habilidades de resolución de problemas: capacidades
			cognitivas, como aprendizaje, memoria, y razonamiento.
			\item Desarrollo socio-individual: Analiza la autorregulación
			emocional, interacción social, autonomía personal y adaptación al
			entorno familiar y comunitario.
		\end{enumerate}
Para cada área del desarrollo infantil se establecerán tres categorías de
clasificación de acuerdo a los valores Z validados por el \asq:
		\begin{itemize}
			\item Desarrollo adecuado: Puntuación dentro o por encima del rango
			esperado para la edad ($\geq$-1 desviación estándar).
			\item Zona de monitoreo: Puntuación ligeramente por debajo del rango
			esperado (-1 a -2 desviaciones estándar).
			\item Riesgo de retraso: Puntuación significativamente por debajo
			del rango esperado ($\leq$-2 desviaciones estándar).
		\end{itemize}
	\item Características sociodemográficas: sexo, edad, nivel educativo de los
padres, lugar de residencia, grupo étnico, número de personas por vivienda,
acceso a servicios básicos (agua potable, electricidad, saneamiento), tipo de
fuente energética para cocinar, método de eliminación de residuos y condición
de vivienda (propia o alquilada).
	\item Condiciones económicas: situación de empleo de los padres, tipo de
empleo y acceso a cobertura de seguridad social.
	\item Interacción familiar: tiempo diario que el padre o tutor dedica al
juego interactivo con el niño y cantidad de juguetes disponibles para el niño.
	\item Exposición a dispositivos electrónicos: tipo de dispositivo
electrónico (celular, televisión, tablet) y duración promedio de uso diario.
	\item Antecedentes médicos: historial de prematuridad, peso al nacer, tipo
de parto, tipo de profesional que atendió el parto, antecedentes de lactancia
materna y alimentación complementaria, estado nutricional actual y
suplementación con micronutrientes.
\end{enumerate}

\section{Hipótesis}
\begin{enumerate}
	\item Hipótesis nula (H0): No existe una asociación significativa entre
	factores sociodemográficos, condiciones económicas, interacción familiar,
	exposición a dispositivos electrónicos, antecedentes médicos perinatales y
	postnatales, y el riesgo en el neurodesarrollo de niños menores de 5 años
	en servicios de atención primaria de Quetzaltenango.
\item Hipótesis alternativa (H1): Existe una asociación significativa entre
	factores sociodemográficos, condiciones económicas, interacción familiar,
	exposición a dispositivos electrónicos, antecedentes médicos perinatales y
	postnatales, y el riesgo en el neurodesarrollo de niños menores de 5 años
	en servicios de atención primaria de Quetzaltenango.
\end{enumerate}

\section{Técnicas de recolección de información e instrumentos de medición}
\begin{enumerate}
	\item Técnicas de recolección de información:
	Para llevar a cabo este estudio de cohorte prospectivo, se implementarán
	las siguientes fases:
	\begin{enumerate}
		\item Fase preliminar (Febrero de 2025):
		Se obtuvieron los permisos correspondientes a las autoridades de salud
		del departamento de Quetzaltenango para acceder a los servicios de
		atención primaria seleccionados. Se determinaron estrategias para
		garantizar la uniformidad en la recolección de los datos entre los
		investigadores.
		\item Fase de reclutamiento inicial (Marzo a Mayo de 2025):
		Se identificarán y reclutarán niños menores de 5 años que cumplan con
		los criterios de inclusión en los servicios de atención primaria
		participantes. Tras obtener el consentimiento informado de los padres o
		tutores, se realizará:
			\begin{itemize}
			\item Evaluación basal del neurodesarrollo mediante la aplicación
			del \asq, seleccionando la versión específica según la edad del
			niño.
			\item Aplicación de un cuestionario estructurado para recolectar
			información sobre factores potencialmente asociados al
			neurodesarrollo.
			\end{itemize}
		\item Fase de seguimiento (Abril a Junio de 2025):
		Se realizará un seguimiento a los 2 meses de la evaluación inicial
		donde:
			\begin{itemize}
			\item Se aplicará nuevamente el \asq\ para evaluar cambios en el
			neurodesarrollo.
			\item Se actualizará la información sobre factores de exposición
			que puedan haber cambiado durante el período de seguimiento.
			\item Se documentarán eventos relevantes ocurridos durante el
			período de seguimiento.
			\end{itemize}
		\item Fase de clasificación y análisis:
		Los resultados longitudinales de cada niño serán evaluados conforme al
		puntaje obtenido en el ASQ-3 en ambos momentos (inicial y seguimiento)
		y clasificados en tres categorías:
			\begin{itemize}
			\item Desarrollo típico: puntaje en el área blanca, indicativo de
			un desarrollo acorde a su edad.
			\item Requiere monitoreo: puntaje en el área gris, señalando
			habilidades ligeramente por debajo del promedio.
			\item Retraso en el desarrollo: puntaje en el área negra,
			sugiriendo la necesidad de intervención especializada.
			\end{itemize}
		Se analizarán las asociaciones entre los factores de exposición
		identificados y los resultados de neurodesarrollo en la evaluación
		inicial, y cambios en el neurodesarrollo durante el seguimiento.
	\end{enumerate}
	\item Instrumentos de recolección de información
	Para este estudio de cohorte prospectivo, se emplearán los siguientes
	instrumentos:
		\begin{itemize}
		\item \asq: Adaptado al idioma español y ajustado por edad. Esta
		herramienta validada de tamizaje del desarrollo identifica riesgos de
		problemas de neurodesarrollo en niños de 2 a 66 meses. Será aplicado
		por los investigadores con información proporcionada por los padres o
		tutores y mediante observación directa de actividades específicas.
		El \asq\ evalúa cinco áreas del desarrollo: comunicación, motricidad
		gruesa, motricidad fina, resolución de problemas, habilidades
		socioindividuales.

		\item Cuestionario de factores de exposición: Instrumento estructurado
		diseñado específicamente para este estudio que recopilará información
		sobre:
				\begin{itemize}
					\item Variables sociodemográficas (edad, sexo, etnia, nivel
					educativo de los padres)
					\item Variables económicas (empleo de los padres, acceso a
					seguridad social)
					\item Variables de interacción familiar (tiempo de juego,
					disponibilidad de juguetes)
					\item Variables médicas (prematuridad, peso al nacer, tipo
					de parto, lactancia, estado nutricional, etc.)
				\end{itemize}
		\end{itemize}
\end{enumerate}

\section{Plan de análisis de datos}
\begin{enumerate}
	\item Preparación de los datos: Los datos en formato físico serán
	digitados para su uso en el software estadístico Rstudio. Se realizará una
	limpieza de los datos para identificar y corregir posibles errores de
	entrada. Los puntajes obtenidos en cada área del desarrollo del \asq\ se
	convertirán a valores estadísticos. 
	
	\item Análisis descriptivo de datos de la cohorte completa: Se calcularán
	frecuencias y porcentajes de los diferentes factores de riesgo presentes en
	la población a estudiar. Se calcularán medidas de tendencia central como
	media, mediana, y desviación estándar de los puntajes del neurodesarrollo.

	\item Análisis comparativo de los resultados del \asq\ de la cohorte
	completa utilizando las siguientes herramientas estadísticas:
		\begin{itemize}
		\item Chi-cuadrado: para determinar si hay asociación significativa
		entre las variables categóricas y riesgo del retraso en el
		neurodesarrollo.
		\item Prueba t de Student: para comparar puntajes del neurodesarrollo
		entre dos grupos diferentes de una misma categoría (por ejemplo, niños
		con bajo peso al nacer versus niños con peso adecuado al nacer)
		\item Análisis de variancia (ANOVA): para comparar medias de puntajes
		del neurodesarrollo en más de dos grupos diferentes de una misma
		categoría y determinar su variación.
		\end{itemize}
	
	\item Presentación de resultados: se elaborarán tablas y gráficos
	apropiados con intervalos utilizando el software Rstudio y paquetes de
	tidyverse para análisis y creación de datos informativos.
\end{enumerate}

\section{Principios éticos en la investigación}
Esta investigación se adherirá a los principios éticos clave, tales como:
\begin{itemize}
	\item Consentimiento informado: explicando claramente los objetivos del
	estudio a los padres o tutores y obteniendo su autorización.

	\item Confidencialidad: los datos se mantendrán anónimos y se utilizarán 
	exclusivamente para fines de investigación.

	\item Beneficencia y no maleficencia: buscando maximizar beneficios
	potenciales sin causar daños a los participantes.
\end{itemize}

	\chapter{Alcances y límites}
Esta investigación se llevará a cabo con niños menores de cinco años que
asisten a los servicios de salud pública en la ciudad de Quetzaltenango, entre
\tiempito. Dado que el neurodesarrollo infantil en este contexto ha sido poco
explorado a nivel nacional, el estudio al ser analítico busca llenar vacíos de
conocimiento y establecer las bases para futuras investigaciones. El objetivo
principal es identificar la incidencia de riesgos en el neurodesarrollo
utilizando el cuestionario “Edades y Etapas 3”, que proporcionará datos
relevantes para mejorar la detección temprana y las estrategias de
intervención, contribuyendo al desarrollo neurológico óptimo de los niños en la
región. Además, se analizarán factores socioeconómicos, demográficos y
ambientales asociados con el riesgo en el neurodesarrollo para generar
información que invite a reflexionar sobre las políticas públicas y los
programas de salud infantil de manera que puedan ser adaptados a las
necesidades locales.

Sin embargo, la investigación tendrá limitaciones. No se incluirán niños con
diagnósticos previos de trastornos del neurodesarrollo o condiciones médicas
que impacten significativamente su desarrollo. Geográficamente, el estudio se
limita a un distrito de Salud en Quetzaltenango, lo que puede restringir la
generalización de los resultados a otras regiones de Guatemala. También existe
la posibilidad de sesgo de selección debido a la participación voluntaria de
los padres, lo que podría afectar la representatividad de la muestra. Por
último, aunque el cuestionario “Edades y Etapas 3” es una herramienta validada,
no proporciona un diagnóstico definitivo de trastornos del neurodesarrollo; los
resultados deben interpretarse como indicadores de riesgo que requieren
evaluación adicional por profesionales especializados.

\printbibliography
\end{document}
