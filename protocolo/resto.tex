\section{Retraso en el neurodesarrollo}
El retraso en el desarrollo se presenta cuando un niño no logra alcanzar los 
hitos del desarrollo esperados en comparación con sus pares de la misma 
población. Este fenómeno surge a partir de alteraciones en diversos dominios 
específicos, como el desarrollo motor grueso y fino, el habla y lenguaje, el 
funcionamiento cognitivo, las habilidades sociales, el desarrollo psicológico, 
la esfera sexual y las actividades de la vida diaria. \cite{DevelopmentalDelay}

El término retraso del desarrollo constituye un descriptor general de un
fenotipo amplio que requiere especificación mediante la determinación cuidadosa
de uno o más elementos vinculados con el área de desarrollo comprometida.
\cite{DevelopmentalDelay} Por otra parte, el concepto de discapacidad del
desarrollo abarca un grupo diverso de alteraciones físicas y mentales
permanentes que afectan negativamente la capacidad del individuo para funcionar
al nivel de sus pares. Estas condiciones comienzan durante la infancia e
interfieren con la movilidad, la adquisición de habilidades de autocuidado, la
comunicación, las competencias sociales, la capacidad general de aprendizaje y
la vida independiente. \cite{Simms2023}

Las discapacidades del desarrollo pueden presentarse de forma aislada, como en 
el caso de un niño con deficiencia visual, o de manera múltiple, como ocurre 
cuando un niño presenta retrasos en el funcionamiento motor, cognitivo, 
lingüístico y social simultáneamente. Con frecuencia, existe un solapamiento 
considerable entre trastornos específicos en términos de las funciones 
afectadas. \cite{Simms2023}

En niños pequeños, los retrasos del desarrollo pueden originarse a partir de 
una amplia variedad de causas, incluyendo la subestimulación ambiental 
temprana, enfermedades físicas crónicas, trastornos neuromusculares, anomalías 
del sistema nervioso central y síndromes genéticos. \cite{Simms2023} La 
disfunción del neurodesarrollo coloca al niño en riesgo de enfrentar desafíos 
en el ámbito del desarrollo, la cognición, las emociones, el comportamiento, 
el funcionamiento psicosocial y la adaptación.

Los niños en edad preescolar con disfunción del neurodesarrollo o ejecutiva 
pueden manifestar retrasos en dominios del desarrollo como el lenguaje, la 
motricidad, el autocuidado, el desarrollo socioemocional y la autorregulación. 
En el caso de los niños en edad escolar, el desarrollo de habilidades
académicas constituye un área de especial atención, siendo esta la etapa en la
que  habitualmente se diagnostican los trastornos del aprendizaje.
\cite{Nelson49}

Aunque no existen estimaciones específicas sobre la prevalencia de la
disfunción del neurodesarrollo, los cálculos generales para los trastornos del
aprendizaje oscilan entre el 5\% y el 10\% o más, con un rango similar
reportado para el TDAH. Es importante señalar que estos trastornos
frecuentemente se presentan de manera simultánea, lo que dificulta tanto su
diagnóstico como su abordaje terapéutico.

\section{Factores determinantes en el neurodesarrollo infantil}
\subsection{Componentes genéticos y epigenéticos}
% Profundizar en cómo las influencias genéticas y epigenéticas
% pueden afectar la trayectoria de desarrollo

\subsection{Entorno familiar y socioeconómico}
% Analizar cómo factores como la dinámica familiar y los recursos
% económicos influyen en el desarrollo infantil

\section{Evaluación del neurodesarrollo infantil en el contexto de salud pública}

\subsection{Cuestionarios Edades y Etapas 3}
Los ``Cuestionarios Edades y Etapas 3'' (ASQ-3, por sus siglas en inglés) son
una herramienta ampliamente utilizada para la detección temprana de riesgos en
el neurodesarrollo infantil. Diseñados inicialmente en la Universidad de Oregon
en los años 1980, estos cuestionarios han evolucionado hasta convertirse en un
instrumento estandarizado y validado para evaluar múltiples dominios del
desarrollo infantil en niños de 1 mes a 66 meses de vida. Su diseño permite que
sean completados por los padres en colaboración con proveedores de salud, lo
que facilita su implementación en diversos contextos.
\cite{Singh2017, ASQ4decades}

La importancia del ASQ-3 radica en su capacidad para identificar de manera
precisa a niños que podrían beneficiarse de una evaluación más detallada o de
intervenciones tempranas en el desarrollo. Este instrumento evalúa cinco áreas
principales del neurodesarrollo: comunicación, motricidad gruesa, motricidad
fina, resolución de problemas y desarrollo personal-social. Cada cuestionario
consta de 30 ítems específicos para la edad del niño, lo cual asegura
pertinencia y sensibilidad en la evaluación. \cite{squires2009ages}

Desde una perspectiva teórica, el ASQ-3 se alinea con el modelo
biopsicosocial del desarrollo infantil, en el que factores biológicos,
ambientales y sociales interactúan complejamente para influir en los
resultados del desarrollo. Este modelo, descrito por Bronfenbrenner, destaca la
importancia de los entornos inmediatos, como la familia y la comunidad, así
como de factores más amplios, como las políticas públicas y las normas
culturales. En este sentido, el ASQ-3 no solo sirve como una herramienta de
cribado, sino también como un catalizador para intervenciones que
potencialmente pueden transformar los entornos más amplios de los niños
evaluados. \cite{Feldman3, Bronfenbrenner2005}

La validez psicométrica del ASQ-3 ha sido extensivamente documentada reportando
una alta fiabilidad en la evaluación-revaluación (92\%), sensibilidad (87.4\%)
y especificidad (95.7\%). Además, su aplicabilidad ha sido comprobada en
contextos multiculturales y socioeconómicos diversos, incluyendo países de
ingresos bajos y medios, donde su uso requiere adaptaciones contextuales debido
a limitaciones en la alfabetización y el acceso a servicios de salud. En tales
contextos, la evidencia sugiere que los resultados del ASQ-3 son más fiables
cuando los cuestionarios se completan con la orientación de un proveedor de
salud. \cite{Vameghi2013-uo, SarmientoCampos2010, Manasyan2023}

\subsubsection{Contexto histórico de los Cuestionarios Edades y Etapas 3}
El desarrollo de los ``Cuestionarios Edades y Etapas'' inició en 1980 en la
Universidad de Oregon, con la finalidad de evaluar el desarrollo neurológico de
los niños con habilidades que los padres son capaces de reconocer e identificar
en casa. En la década de 1980 a 1990 las doctoras Diane Bricker y Jane Squires
realizaron una búsqueda exhaustiva de conjuntos de habilidades fáciles de ser
elicitadas u observadas por los padres de los niños. En 1995 se publicó una
serie de 8 cuestionarios que evaluaban niños hasta los 48 meses de edad.
\cite{ASQ4decades}

En 1996 se iniciaron estudios de validez, fiabilidad y utilidad de la primera
edición del ``Cuestionario Edades y Etapas''. Para el año 1997 y 1998 se los
estudios determinan las propiedades psicométricas del instrumento de tamizaje.
\cite{ASQ4decades}

En 1999 se publica la segunda edición de los cuestionarios revisados y
extendidos para cubrir hasta la edad de 60 meses. En 2000 se inician
investigaciones a nivel internacional en Finlandia, Noruega, China, Portugal y
Brasil. En 2001 la Academia Americana de Pediatría recomienda el ``Cuestionario
Edades y Etapas'' como una herramienta con buenas propiedades psicométricas,
incluyendo sensibilidad, especificidad, validez y fiabilidad adecuadas, y
estandarizada en diferentes poblaciones. \cite{Pediatrics2001}

Para 2004 se inicia una recolección de datos para la tercera edición del
cuestionario y durante 4 años se recolectan 18,000 cuestionarios de niños de 50
estados de territorios estadounidenses. \cite{ASQ4decades}

Para 2006 la Academia Americana de Pediatría revisa su política de tamizaje del
desarrollo con un algoritmo que incluye cribado a los 9, 18 y 30 meses en
chequeos médicos de rutina y recomienda el ``Cuestionario Edades y Etapas''
como una de las principales herramientas para el tamizaje.
\cite{Pediatrics2006}
 
En 2009 la tercera edición de los cuestionarios es publicada, incluyendo nuevos
valores de estandarización, y puntajes revisados para catalogar a los niños en
áreas de riesgo o de monitoreo. Esta tercera edición se ajusta a los resultados
de 18,000 cuestionarios obtenidos de diferentes contextos socioeconómicos y
lugares en los Estados Unidos de América. \cite{ASQ4decades}

En los Estados Unidos de América es ampliamente utilizada en programas de
chequeo médico de rutina por recomendaciones de la Academia Americana de
Pediatría de realizar por lo menos un tamizaje del desarrollo a las edades de
9, 18 y 30 meses de edad. Otros programas como ``Early Head Start'', ``Help Me
Grow'', ``Child Find'', ``Parents as Teachers'', y programas locales de
condados realizan tamizajes y seguimientos del neurodesarrollo con el
``Cuestionario Edades y Etapas 3'' con mayor frecuencia. \cite{ASQWorld}

\subsubsection{Viabilidad de los Cuestionarios Edades y Etapas 3 en Guatemala}
El uso de los ``Cuestionarios Edades y Etapas 3'' (ASQ-3) en Guatemala presenta
una oportunidad significativa para abordar las disparidades en la detección y
el manejo temprano de problemas del neurodesarrollo infantil. Esta herramienta,
con una fiabilidad de evaluación-reevaluación del 92\%, sensibilidad del 87.4\%
y especificidad del 95.7\%, ha demostrado su robustez psicométrica en diversos
contextos culturales y socioeconómicos, incluyendo países de ingresos bajos y
medios. Sin embargo, la implementación del ASQ-3 en Guatemala requiere
consideraciones específicas dadas las particularidades del sistema de salud y
el entorno sociocultural del país. \cite{Vameghi2013-uo, Manasyan2023}

Una característica clave del ASQ-3 es su enfoque centrado en los padres,
quienes completan los cuestionarios en colaboración con proveedores de salud.
Sin embargo, en contextos como Guatemala, donde la alfabetización funcional y
el acceso a servicios de salud son limitados, la evidencia sugiere que los
resultados son más confiables cuando los cuestionarios se completan con el
apoyo de personal capacitado. Esto resalta la importancia de integrar el ASQ-3
en programas comunitarios y de atención primaria, donde los proveedores de
salud puedan desempeñar un rol clave en la interpretación de resultados y el
seguimiento de casos. \cite{Manasyan2023, Colbert2021}
