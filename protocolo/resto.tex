\section{Trastornos del neurodesarrollo y factores de riesgo}
According to the Diagnostic and Statistical Manual of Mental Disorders, \cite{DSM5TR}
The neurodevelopmental disorders are a group of conditions with onset in the developmental period. The disorders typically manifest early in development, often before the child enters school, and are characterized by developmental deficits or differences in brain processes that produce impairments of personal, social, academic, or occupational functioning. The range of developmental deficits or differences varies from very specific limitations of learning or control of executive functions to global impairments of social skills or intellectual ability. Once thought to be categorically defined, more recent dimensional approaches to measurement of the symptoms demonstrate a range of severity, often without a very clear boundary with typical development. Diagnosis of a disorder thus requires the presence of both symptoms and impaired function.

The neurodevelopmental disorders frequently co-occur with one another; for example, individuals with autism spectrum disorder often have intellectual developmental disorder (intellectual disability), and many children with attention-deficit/hyperactivity disorder (ADHD) also have a specific learning disorder. The neurodevelopmental disorders also frequently co-occur with other mental and behavioral disorders with onset in childhood (e.g., communication disorders and autism spectrum disorder may be associated with anxiety disorders; ADHD with oppositional defiant disorder; tics with obsessive-compulsive disorder).
\cite{DSM5TR}

Developmental delay is defined as a developmental quotient of less than 70\%
within a given developmental domain, including gross motor and fine motor,
expressive language, receptive language, and social/adaptive behaviors. The
developmental quotient is the developmental age divided by the child’s
chronologic age (adjusted for prematurity). \cite{vanKarnebeek2018}

The DSM-5TR classifies neurodevelopmental disorders in five categories:
	\begin{itemize}
		\item Intellectual Developmental Disorders
		\item Communication Disorders
		\item Autism Spectrum Disorder
		\item Attention-Deficit/Hyperactivity Disorder
		\item Specific Learning Disorder
		\item Motor Disorders
	\end{itemize}

\subsection{Intellectual Developmental Disorders}
\subsubsection{Definition}
ID is characterized by significant limitations in intellectual functioning and
in adaptive behavior that begin before age 18 years and are expressed
in conceptual, social, and practical adaptive skills. The impairments of
ID extend beyond what is measured on a standardized test of intelligence and
must take into account the context of an individual’s typical
environment and their cultural and linguistic backgrounds. \cite{Simms2023}

Adaptive functioning includes three broad domains: conceptual, social, and
practical. The conceptual domain involves academic competence, the
acquisition of practical knowledge, and judgment in novel situations.
The social domain involves awareness of others’ thoughts and feel-
ings, empathy, friendships, and social judgment. The practical domain
involves the ability to manage one’s own affairs, including school and
work responsibilities, money management, and recreation. \cite{Simms2023}

Children with ID have a nonprogressive disorder. Individuals with ID will
acquire new developmental milestones over time, although at a slower
rate than unaffected children. A slowing trajectory is not uncom-
mon as individuals with ID get older. \cite{Nelson56}

\subsubsection{Epidemiology}
The worldwide prevalence of intellectual disability is estimated at 1\% to 3\%.
The prevalence of ID has been estimated to be approximately 16.4 per 1,000
persons in low-income countries, approximately 15.9 per 1,000 for middle-income
countries, and approximately 9.2 per 1,000 in high income countries.
\cite{vanKarnebeek2018, Nelson56}

\subsubsection{Risk factors}
Numerous identified causes of ID may occur prenatally, during delivery,
postnatally, or later in childhood. These include infection, trauma,
prematurity, hypoxia-­ischemia, toxic exposures, metabolic dysfunction,
endocrine abnormalities, malnutrition, and genetic abnormalities. Most
people with ID will not have a readily identifiable underlying diagnosis based
on prenatal or perinatal history or dysmorphology, meriting
further medical and genetic evaluation. \cite{Nelson 56}

Mild and more severe forms of ID have different but overlapping risk
factors and etiologies. Nongenetic risk factors that are often associated
with mild ID include low socioeconomic status, low maternal education levels,
residence in a developing country, malnutrition, and poor
access to healthcare. The most common biologic causes of, or risk factors for,
mild ID include intrauterine growth restriction; prematurity;
perinatal insults; intrauterine exposure to drugs of abuse (including
alcohol); postnatal exposure to neurotoxic substances (including lead);
some sex chromosomal abnormalities; and some genetic syndromes
with multiple, major, or minor congenital anomalies (e.g., 22q11 deletion
syndrome, sex chromosomal abnormalities, Noonan syndrome). \cite{Nelson56}

In children with more severe ID, a biologic cause (usually with prenatal onset)
can be identified in about three fourths of all cases. Causes
include chromosomal (e.g., Down, Wolf-­Hirschhorn, and deletion 1p36
syndromes) and other genetic and epigenetic disorders (e.g., fragile X,
Rett, and Angelman syndromes), abnormalities of brain development
(e.g., lissencephaly), and inborn errors of metabolism and mitochondrial
disorders (e.g., mucopolysaccharidoses, mitochondrial respiratory
chain complex disorders). Nonsyndromic severe ID may be
a result of inherited or de novo gene mutations, as well as microdeletions or
microduplications. \cite{Nelson56}

\subsubsection{Diagnosis}
The formal diagnosis of ID requires the administration of individual
tests of intelligence and adaptive functioning.

The Bayley Scales of Infant and Toddler Development, Fourth Edition, the most
used infant intelligence test, provides an assessment
of cognitive, language, motor, behavior, social-­emotional, and general
adaptive abilities between 16 days and 42 months of age. \cite{Nelson56}

The most used intelligence tests for children older than 3 years
are the Wechsler Scales. Among children who have marked language or verbal
limitations, tests like the Differential Ability Scales-II (DAS-­II) or the
Leiter International Performance Scale, Third Edition (Leiter-3) may be used
to optimally capture nonverbal performance skills. \cite{Nelson56}

\subsection{Communication Disorders}
Disorders of communication include deficits in language, speech, and communication. Speech is the expressive production of sounds and includes an individual’s articulation, fluency, voice, and resonance quality. Language includes the form, function, and use of a conventional system of symbols in a rule-governed manner for communication. Communication includes any verbal or nonverbal behavior (whether intentional or unintentional) that has the potential to influence the behavior, ideas, or attitudes of another individual. \cite{DSM5TR}

\subsubsection{Language Disorder}
The essential features of language disorder are difficulties in the acquisition
and use of language due to deficits in the comprehension or production of
vocabulary, grammar, sentence structure, and discourse. The language deficits
are evident in spoken communication, written communication, or sign language.
\cite{DSM5TR}

Primary disorders of speech and language development are significant
difficulties found in the absence of major cognitive, sensory, or motor
dysfunction. The criterion for language delay is performance at least
1.5 to 2 SD below the population mean on standardized tests
of speech or language. \cite{Feldman44}

Children with language impairment may have
significant difficulty in higher-level language skills,
reasoning skills (e.g., drawing correct inferences and conclusions),
the ability to take another person’s perspective, and the ability to paraphrase
and rephrase. Some children with language impairment show
difficulties with social interaction because social interactions are often
mediated by verbal language. Young children with language impairment may
interact more successfully with older children or adults, who
can adapt their communication to match the child’s level of function,
than with peers. \cite{Nelson53}

\paragraph{Epidemiology}
Disorders of language and speech are highly prevalent.
Approximately 16\% of children show clinically significant
delays at age 2; approximately half of those children remain
delayed to kindergarten entry. Speech and language impair-
ment is the most common condition in preschool years estab-
lishing eligibility for special education services. Language and
speech disorders may occur in isolation, together, and in con-
junction with other delays or disorders. \cite{Feldman44}

\paragraph{Risk Factors}
Genetic factors appear to play a major role in influencing how children
learn to talk. A family history may identify current or past speech or
language problems in up to 30\% of first-degree relatives of proband
children. Concordance rate for low language test scores and/or a his-
tory of speech therapy within twin pairs is about 50\% in dizygotic
pairs and 90\% in monozygotic pairs. Consistent pathogenic genetic
variations have not been identified. Instead, multiple genetic regions
and epigenetic changes may result in heterogeneous genetic pathways
causing language disorders. Environmental, hormonal, and nutritional factors
may exert epigenetic influences by dysregulating gene expression. \cite{Nelson53}

Children raised in orphanages typically experience language and speech
disorders. Children who have experienced child abuse and child neglect also
frequently develop language disorders. For children living in poverty, language
delays may be related to poor language nutrition and other challenges, such as
poor nutrition and high levels of stress. \cite{Feldman44}

\paragraph{Diagnosis}
Accurate assessment of infants and toddlers is challenging because of the low
frequency of verbal output and the difficulty that young children have in
cooperating with clinicians. Informal observations, parent interview tools, and
natural assessments play an important role in the evaluation of young children.
Formal assessments become more central to evaluation as the child reaches
preschool age and beyond. \cite{Nelson53}

\paragraph{Comorbidities}
Language disorder may be associated with other neurodevelopmental disorders in
terms of specific learning disorder (literacy and numeracy), intellectual
developmental disorder, attention-deficit/hyperactivity disorder, autism
spectrum disorder, and developmental coordination disorder. \cite{Feldman44}

\subsubsection{Speech Sound Disorders}
A speech sound disorder represents impairment in the ability
to produce sounds of the words of the language. A primary
symptom of speech impairment may be unintelligible speech.
Speech disorders are described in terms of the characteristics
of the speech sound errors or the cause of the problem. Often
an underlying cause for the speech sound disorder cannot be
identified. \cite{Feldman44}

\paragraph{Articulation Disorders}
The inability to produce speech sounds correctly is referred to
as an articulation disorder. Children with articulation disor-
ders typically exhibit errors on a small subset of sounds (e.g., /r, l, s/). In
most cases the cause of an articulation disorder is
unknown; they are presumed to be the result of mislearning.
One known cause of articulation disorders is permanent
bilateral mild to moderate hearing loss. \cite{Feldman44}

\paragraph{Phonologic Disorders}
When a child shows speech errors based on patterns or implicit
rules despite the ability to produce the same sounds correctly
in other contexts, the condition is labeled a phonologic disor-
der. Un niño que omite las
últimas consonantes puede decir ``feli'' en lugar de ``feliz'' y ``canta'' en lugar
de ``cantar'', pero es probable que no tenga ningún problema al decir palabras
como ``zombi'' o ``rana''. The child is applying an
incorrect rule rather than showing an inability to produce the sound
correctly. \cite{Feldman44}

Children with phonologic errors typically
have moderate to severe deficits in speech skills. 

\paragraph{Anatomic Disorders}
Ankyloglossia, or tongue tie, is a common congenital anomaly that has been
assumed to affect speech. However, ankyloglossia does not typically cause
speech impairment. While
tongue tie may be significant at birth, its severity decreases
over time as oral structures grow. Moreover, speech sounds
can be produced with minimal tongue elevation. Cutting
the tongue frenulum typically does not improve speech. An
exception may be children with cerebral palsy whose ankyloglossia is related to
neurologic impairments. \cite{Feldman44}

Children with cleft palate are at high risk for phonologic
disorders and language deficits. Even after repair of an iso-
lated cleft palate, they may exhibit unusual or idiosyncratic
patterns of articulation. Velopharyngeal structures function
abnormally, resulting in an inability to generate adequate
intraoral air pressure for the production of consonants. \cite{Feldman44}

Some children have isolated velopharyngeal insufficiency
for unknown reasons. These children are at risk for speech
sound disorders similar to those of children with cleft palate. \cite{Feldman44}

\paragraph{Neurologic Disorders}
Dysarthria is a speech disorder associated with neuromo-
tor disorders, such as cerebral palsy. High muscle tone, poor
coordination of motor movements, and poor coordination
of respiration and sound production result in slow mus-
cular movements and limited range of motion. Dysarthric
speech has a slurred and strained quality, affecting accuracy
of speech sound production, rate, pitch, and intonation. \cite{Feldman44}

Childhood apraxia of speech, known previously as devel-
opmental verbal apraxia or dyspraxia, is a condition in which
children have difficulty with the controlled production of
speech sounds. The presumed etiology is neurologic in ori-
gin, although anatomic lesions are not usually found.
Children make errors in producing vowels and consonants
and show enormous variability in how they produce pho-
nemes and in loudness. The inconsistency in production and
in errors makes interpretation of their speech very challenging. \cite{Feldman44}

\paragraph{Stuttering}
Stuttering is the most common cause of significant dysfluency, manifested by
repetition of sounds and syllables and
prolongation of vowels or consonants made with a continuous
airflow. Stuttering is often accompanied by inappropriate pauses, repetitive
facial expressions, or other behavioral routines. Stuttering is now considered
a neurodevelopmental disorder, characterized by atypical development of neural
networks involved in speech motor planning and execution.
Stuttering runs in families, suggestive of genetic contributions. \cite{Feldman44}

\subsection{Autism Spectrum Disorder}
Autism spectrum disorder (ASD) represents a group of lifelong
neurodevelopmental disorders emerging during early childhood and it is
characterized by impaired social communication and interaction accompanied by
restricted and repetitive behaviors.

\subsubsection{Historical Definitions of ASD}
The first clinical description of six children with a constellation of traits
easily recognizable as what is currently known as ASD was published by Russian
psychiatrist Grunya Sukhareva in 1925. \cite{Myers2025}

In 1943 Leo Kanner eloquently described 11 children with innate ``autistic
disturbances of affective contact'' characterized by profoundly deficient
social engagement and interaction; communication disturbance ranging from
mutism to echolalia, pronoun reversal, and literalness; and unusual
interactions with the environment, including an ``anxiously obsessive desire
for the maintenance of sameness'' that was not explained by general cognitive
impairment. \cite{Myers2025}

The next year, Hans Asperger contributed a report of four boys with a similar
constellation of characteristics constituting ``autistic psychopathy'',
including lack of reciprocity in social interactions and a restricted range of
intense and unusual interests, but normal or precocious language acquisition,
with above-average linguistic skills but subtle abnormalities of verbal and
nonverbal communication and motor clumsiness. \cite{Myers2025}

\subsubsection{Current Definition}
Almost 80 years after the seminal papers by Kanner and Asperger, and nearly a
century after the initial prescient description by Sukhareva, the ASD
diagnostic criteria remain remarkably consistent with the original observations
of these astute clinicians. \cite{Myers2025}

Before DSM-5, five overlapping disorders captured the spectrum: autistic
disorder, Asperger disorder, childhood disintegrative disorder, Rett syndrome,
and pervasive developmental disorder not otherwise specified.
\cite{Boland2021-by}

DSM-5 and ICD-11 are well-aligned in their conceptualization of ASD as a single
diagnosis defined by clinically significant, persistent deficits in social
communication and interaction and atypically restricted and repetitive
behaviors and interests that cause significant impairment in adaptive
functioning. The specific manifestations of the core deficits vary with age,
language and intellectual ability, and disorder severity. Symptoms begin in
the developmental period but may not become fully manifest until social demands
exceed limited capacities. The deficits in social communication and interaction
must be present to a degree that is outside of the expected range of
functioning for the individual’s age and level of intellectual development.
\cite{DSM5TR, Myers2025}

\subsubsection{Epidemiology}
The overall prevalence of ASD in Europe, Asia, and the United States ranges
from 2 to 25 per 1000, or approximately 1 in 40 to 1 in 500. The prevalence of
ASD has increased over time, particularly since the late 1990s. Systematic
reviews of epidemiologic studies suggest that changes in case definition and
increased awareness account for much of the apparent increase.
There is a 4:1 male predominance. \cite{AutismUpToDate, Nelson58} 

\subsubsection{Etiology and Risk Factors}
The pathogenesis of ASD is incompletely understood. The general consensus is
that ASD is caused by genetic factors that alter brain development,
specifically neural connectivity, thereby affecting social communication
development and leading to restricted interests and repetitive behaviors.
This consensus is supported by the ``epigenetic theory'', in which an abnormal
gene is ``turned on'' early in fetal development and affects the expression of
other genes without changing their primary DNA sequence. \cite{AutismUpToDate}

Given the complexity of ASD and the diversity of clinical manifestations, it is
likely that interactions between multiple genes or gene combinations are
responsible for ASD and that epigenetic factors and exposure to environmental
modifiers contribute to the variable expression. \cite{AutismUpToDate}

A strong genetic contribution to the development of ASD is supported by the
unequal sex distribution, increased prevalence in siblings, high concordance in
monozygotic twins, and increased risk of ASD with increased relatedness.
\cite{AutismUpToDate}

A higher-than-expected incidence of prenatal and perinatal complications seems
to occur in infants who later have autism spectrum disorder. The most
significant prenatal factors associated with autism spectrum disorder in the
offspring are advanced maternal and paternal age at birth, maternal gestational
bleeding, gestational diabetes, and first-born baby. Perinatal risk factors for
autism spectrum disorder include umbilical cord complications, birth trauma,
fetal distress, small for gestational age, low birth weight, low 5-minute Apgar
score, congenital malformation, ABO blood group system or Rh factor
incompatibility and hyperbilirubinemia. \cite{Boland2021-by}

There is also possible evidence for environmental contributions to ASD.
Population-level associations have been investigated for environmental toxins
such as organophosphates, pesticides, air pollution, and volatile organic
compounds. An epigenetic model is considered one explanation for the etiology;
individuals with genetic vulnerability may be more sensitive to environmental
factors influencing early brain development. \cite{Nelson58}

\subsubsection{Diagnosis}
The diagnostic criteria in the DSM-5 focus on symptoms in two primary domains:
	\begin{enumerate}
		\item Social communication and social interaction
		\item Restricted interests and repetitive behaviors
	\end{enumerate}
To meet criteria for ASD, the symptoms need to have been present since the
early developmental period, significantly affect functioning, and not be
better explained by the diagnosis of intellectual disability or global
developmental delay. \cite{DSM5TR}

The gold standard test is the Autism Diagnostic Observation Schedule (ADOS),
which is available for individuals 12 months of age through adulthood and
requires about an hour to administer. The person administering the exam must
have special training and certification in the ADOS. \cite{Koth2023}

\subsection{Attention-Deficit/Hyperactivity Disorder}
ADHD is a neuropsychiatric condition affecting preschoolers, children, adolescents, and adults around the world, characterized by inattention, including increased distractibility and difficulty sustainin
attention; poor impulse control and decreased self-inhibitory capacity; and motor overactivity and restlessness. \cite{Boland2021-by, Nelson50}

ADHD can have its onset in infancy, although it is rarely recognized until a child is at least toddler age. More commonly, infants with ADHD are active in the crib, sleep little, and cry a great deal. In school, children with ADHD may attack a test rapidly but may answer only the first two questions. They may be unable to wait to be called on in school and may respond before everyone else. At home, caregivers cannot put them off for even a minute. Impulsiveness and an inability to delay gratification are characteristic. Children with ADHD are often susceptible to accidents. \cite{Boland2021-by}

The most cited characteristics of children with ADHD, in order of frequency, are hyperactivity, attention deficit (short attention span, distractibility, perseveration, failure to finish tasks, inattention, poor concentration), impulsivity (action before thought, abrupt shifts in activity, lack of organization, jumping up in class), memory and thinking deficits, specific learning disabilities, and speech and hearing deficits. Associated features often include perceptual-motor impairment, emotional lability, and developmental coordination disorder. A significant percentage of children with ADHD show behavioral symptoms of aggression and defiance. School difficulties, both learning and behavioral, commonly exist with ADHD. Comorbid communication disorders or learning disorders that hamper the acquisition, retention, and display of knowledge complicate the course of ADHD. \cite{Boland2021-by}

\subsubsection{Epidemiology}
ADHD affects up to 5 to 8 percent of school-age children, with 60 to 85 percent of those diagnosed as children continuing to meet criteria for the disorder in adolescence, and up to 60 percent continuing to be symptomatic into adulthood. Children, adolescents, and adults with ADHD often have significant impairment in academic functioning as well as in social and interpersonal situations. ADHD is frequently associated with comorbid disorders, including learning disorders, anxiety disorders, mood disorders, and disruptive behavior disorders. \cite{Boland2021-by}

ADHD is more common in males than females (male to female ratio 4:1 for the
predominantly hyperactive-impulsive presentation and 2:1 for the predominantly
inattentive presentation). \cite{ADHDUpToDate}

\subsubsection{Etiology and risk factors}
Data suggest that the etiology of ADHD is mainly genetic, with a heritability of approximately 75 percent. ADHD symptoms are the product of complex interactions of neuroanatomical and neurochemical systems evidenced by data from twin and adoption family genetic studies, dopamine transport gene studies, neuroimaging studies, and neurotransmitter data. Most children with ADHD have no evidence of gross structural damage in the CNS. In some cases, contributory factors for ADHD may include prenatal toxic exposures, prematurity, and prenatal mechanical insult to the fetal nervous system. Some have suggested that food additives, colorings, preservatives, and sugar are possible contributing causes of hyperactive behavior; however, studies have not confirmed these theories. No research has established artificial food coloring nor sugar as causes of ADHD. There is no clear evidence that omega-3 fatty acids are beneficial in the treatment of ADHD. \cite{Boland2021-by, ADHDUpToDate, Nelson50}

\subsubsection{Diagnosis}
The current edition of the DSM-5 was published in 2013, and it presents the diagnosis of ADHD as a threshold diagnosis in which a child must have at least six of the described symptoms of inattention or impulsivity/hyperactivity with
several symptoms occurring before the age of 12 years, persisting for at least 6 months, occurring across two or more
environments, and reducing the quality of social, academic, or occupational functioning, while not being better
accounted for by another disorder. \cite{Lazar2025}

\subsection{Specific Learning Disorder}
Specific learning disorder in youth is a neurodevelopmental disorder produced by the interactions of heritable and environmental factors that influence the brain’s ability to perceive or process verbal and nonverbal information efficiently. Children with the disorder have persistent difficulty learning academic skills in reading, written expression, or mathematics, beginning in early childhood, which is inconsistent with the overall intellectual ability of a child. \cite{Boland2021-by}

\subsubsection{Epidemiology}
An estimated 5\% 15\% of school children struggle with a developmental disorder
that disrupts their education by impairing their ability to learn skills specific to the academic domains of reading, writing, and/or math. \cite{Frierson2025}

SLD ranges in severity from mild (e.g., a single area of deficit affecting only one academic domain that may require
an accommodation) to severe (e.g., learning deficits affecting all three domains that may require extensive special edu-
cation support). Reported estimates of male to female ratio range from nearly equal (1.15:1) to strong male predomi-
nance (5:1). The strong male predominance is particularly related to more severe symptoms \cite{Frierson2025}

\subsubsection{Risk factors}
SLD is highly heritable. An individual with a family history of SLD has
quadruple the risk of having SLD. The causes of these disorders are
multifactorial, including genetic, maturational, cognitive, emotional,
educational, and socioeconomic factors. Prematurity and very low birth weight
are also risk factors for specific learning disorder \cite{Frierson2025, Boland2021-by}

\subsubsection{Classification}
\paragraph{Specific Learning Disorder with Impairment in Reading}
Reading impairment is present in up to 75 percent of children and adolescents
with a specific learning disorder. Students who have learning problems in other
academic areas most commonly experience difficulties with reading as well.
\cite{Boland2021-by}

Reading impairment is characterized by difficulty in recognizing words, slow
and inaccurate reading, poor comprehension, and difficulties with spelling.
Reading impairment is often comorbid with other disorders in children,
particularly ADHD. \cite{Boland2021-by}

\paragraph{Specific Learning Disorder with Impairment in Mathematics}
Children with mathematics difficulties have difficulty learning and remembering numerals, cannot remember basic facts about numbers, and are slow and inaccurate in computation. There are four groups of skills for which children with this disorder have poor achievement: linguistic skills (those related to understanding mathematical terms and converting written problems into mathematical symbols), perceptual skills (the ability to recognize and understand symbols and order clusters of numbers), mathematical skills (basic addition, subtraction, multiplication, division, and following the sequencing of basic operations), and attentional skills (copying figures correctly and observing operational symbols correctly). \cite{Boland2021-by}

\paragraph{Specific Learning Disorder with Impairment in Written Expression}
Written expression is the most complex skill acquired to convey an understanding of language and to express thoughts and ideas. Writing skills are highly correlated with reading for most children; however, for some youth, reading comprehension may far surpass their ability to express complex thoughts. Written expression, in some cases, is a sensitive index of more subtle deficits in language usage that typically are not detected by standardized reading and language tests.
Deficits in written expression include writing skills that are significantly below the expected level for a child’s age and education. Such deficits impair the child’s academic performance and writing in everyday life. Components of writing disorder include poor spelling, errors in grammar and punctuation, and poor handwriting. Spelling errors are among the most common difficulties for a child with a writing disorder. Spelling mistakes are most often phonetic errors; that is, an erroneous spelling that sounds like the correct spelling. \cite{Boland2021-by}

\subsection{Motor Disorders}


\section{Evaluación del neurodesarrollo infantil en el contexto de salud pública}


