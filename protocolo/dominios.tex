\section{Hitos del desarrollo infantil y teorías del desarrollo humano}

Child development can be tracked by the child’s developmental progress in
particular domains, such as gross motor, fine motor, social, emotional,
language, and cognition. Within each of these categories are  developmental
sequences of changes leading up to particular attainments. Development in the
gross motor domain, from rolling to creeping to independent walking, are clear.
Others, such as the line leading to the development of conscience, are subtler.
The concept of a developmental line implies that a child passes through
successive stages. Several developmental theories are based on stages as
qualitatively different epochs in the development of emotion and cognition. In
contrast, behavioral theories rely less on qualitative change and more on the
gradual modification of behavior and accumulation of competence.
\cite{Nelson19}

The neurodevelopmental processes that are critical to a child’s successful
functioning may best be understood as falling within core neurodevelopmental
domains that are highly integrated.

Although neurodevelopment follows a predictable course, it is important to
understand that intrinsic and extrinsic forces produce individual variation,
making each child’s developmental path unique. Intrinsic influences include
genetically determined attributes (eg, physical characteristics, temperament)
as well as the child’s overall state of wellness. Extrinsic influences during
infancy and childhood originate primarily from the family. Parent and sibling
personalities, the nurturing methods used by caregivers, the cultural
environment, and the family’s socioeconomic status with its effect on resources
of time and money all play a role in the development of children.
\cite{Gerber2010}

\subsection {Desarrollo motor grueso}
Gross motor skills are large general movements that require development of
muscle tone and involve large muscle activities. For example, controlling one’s
posture – the ability to sit upright and/or stand – is a gross motor skill that
is foundational in the development of other gross motor skills like crawling
and walking. \cite{Lorentz2021}

The ultimate goal of gross motor development is to gain independent and
volitional movement. During gestation, primitive reflexes develop and persist
for several months after birth to prepare the infant for the acquisition of
specific skills. These brainstem and spinal reflexes are stereotypic movements
generated in response to specific sensory stimuli. Examples include the Moro,
asymmetric tonic neck, and positive support reflexes. As the central nervous
system matures, the reflexes are inhibited to allow the infant to make
purposeful movements. For example, during the time when the ATNR persists, an
infant is unable to roll from back to front, bring the hands to midline, or
reach for objects. This reflex disappears between 4 and 6 months of age, the
same time that these skills begin to emerge. The Moro reflex interferes with
head control and sitting equilibrium. As this reflex lessens and disappears by
6 months of age, the infant gains progressive stability in a seated position.
\cite{Gerber2010}

In addition to primitive reflexes, postural reactions, such as righting and
protection responses, also begin to develop after birth. These reactions,
mediated at the midbrain level, interact with each other and work toward the
establishment of normal head and body relationship in space. Protective
extension, for example, allows the infant to catch him- or herself when falling
forward, sideways, or backwards. These reactions develop between 6 and 9
months, the same time that an infant learns to move into a seated position and
then to hands and knees. Soon afterward, higher cortical centers mediate the
development of equilibrium responses and permit the infant to pull to stand by
9 months of age and begin walking by 12 months. Additional equilibrium
responses develop during the second year after birth to allow for more complex
bipedal movements, such as moving backward, running, and jumping.
\cite{Gerber2010}

During the first postnatal year, an infant thus moves from lying prone, to
rolling over, to getting to hands and knees, and ultimately to coming to a
seated position or pulling to stand. \cite{Gerber2010}

It is important to note that crawling is not a prerequisite to walking;
pulling to stand is the skill infants must develop before they take their first
steps. The ultimate goal of this timeframe is to develop skills that allow for
independent movement and freedom to use the hands to explore, manipulate, and
learn from the environment. Gross motor development in subsequent years
consists of refinements in balance, coordination, speed, and strength.
\cite{Gerber2010}

Simultaneous use of both arms or legs occurs after successful use of each limb
independently. At age 2 years, a child can kick a ball, jump with two feet off
the floor, and throw a big ball overhand. Milestones for succeeding ages
reflect progress in the length of time, number of repetitions, or the distance
each task can be performed successfully. By the time a child starts school, he
or she is able to perform multiple complex gross motor tasks simultaneously
(such as pedaling, maintaining balance, and steering while on a bicycle).
\cite{Gerber2010}

\subsection {Desarrollo motor fino}
Fine motor skills refers to small movements of fingers, toes, and wrists to
fulfil tasks like picking up a small object with the hand, holding a rattle, or
lifting a spoon from the floor. \cite{Panda2021}

Fine motor skills relate to the use of the upper extremities to engage and 
manipulate the environment. They arenecessary for a person to perform self-help 
tasks, to play, and to accomplish work. Like all developmental streams, fine
motor milestones do not proceed in isolation but depend on other areas of
development, including gross motor, cognitive, and visual perceptual skills. At
first, the upper extremities play an important role in balance and mobility.
Hands are used for support, first in the prone position and then in sitting.
Arms help with rolling over, then crawling, then pulling to stand. Infants
begin to use their hands to explore, even when in the supine position.
When gross motor skills have developed such that the
infant is more stable in upright positions and can move
into them easily, the hands are free for more purposeful
exploration. \cite{Gerber2010}

At birth, infants do not have any apparent voluntary use of their hands. They
open and close them in response to touch and other stimuli, but movement
otherwise is dominated by a primitive grasp reflex. Because of this, infants
spend the first 3 months after birth “contacting” objects with their eyes
rather than their hands, fixating on faces and objects and then visually
tracking objects. Gradually, they start to reach clumsily and bring their hands
together. As the primitive reflexes decrease, infants begin to prehend objects
voluntarily, first using the entire palm toward the ulnar side (5 months) and
then predominantly using the radial aspect of the palm (7 months). At the same
time, infants learn to release objects voluntarily. In the presence of a strong
grasp reflex, objects must be removed forcibly from an infant’s grasp or drop
involuntarily from the hand. Voluntary release is seen as the infant learns to
transfer objects from one hand to the other, first using the mouth as an
intermediate stage (5 months) and then directly hand-to-hand (6 months).
\cite{Gerber2010}


Between 6 months and 12 months of age, the grasp evolves to allow for
prehension of objects of different shapes and sizes (Fig. 7). The thumb becomes
more involved to grasp objects, using all four fingers against the thumb (a
“scissors” grasp) at 8 months, and eventually to just two fingers and thumb
(radial digital grasp) at 9 months. A pincer grasp emerges as the ulnar fingers
are inhibited while slightly extending and supinating the wrist. Voluntary
release is awkward at first, with all fingers extended. By 10 months of age,
infants can release a cube into a container or drop things onto the floor.
Object permanence reinforces the desire to practice this skill over and over.
Intrinsic muscle control develops to allow the isolation of the index finger,
and infants will poke their fingers into small holes for exploration. By 12
months of age, most infants enjoy putting things into containers and dumping
them out repeatedly. They also can pick up small pieces of food with a mature
pincer grasp and bring them to their mouths. \cite{Gerber2010}

As infants move into their second year, their mastery of the reach, grasp, and
release allows them to start using objects as tools. Fine motor development
becomes more closely associated with cognitive and adaptive development, with
the infant knowing both what he or she wants to do and how he or she can
accomplish it. Intrinsic muscle refinement allows for holding flat objects,
such as crackers or cookies. By 15 months of age, voluntary release has
developed further to enable stacking of three to four blocks and releasing
small objects into containers. The child starts to adjust objects after
grasping to use them properly, such as picking up a crayon and adjusting it to
scribble spontaneously (18 months of age) and adjusting a spoon to use it
consistently for eating (20 months of age). \cite{Gerber2010}

In subsequent years, fine motor skills are refined further to draw, explore,
problem-solve, create, and perform self-help tasks. By age 2 years, children
can create a six-block tower, feed themselves with a spoon and fork, remove
clothing, and grasp and turn a door knob. They have sufficient control of a
crayon to imitate both vertical and horizontal lines. In-hand manipulation
skills permit them to rotate objects, such as unscrewing a small bottle cap or
reorienting a puzzle piece before putting it in place. They are able to wash
and dry their hands. By 36 months of age, they can draw a circle, put on shoes,
and stack 10 blocks. They make snips with scissors by alternating between
full-finger extension and flexion. Their grasp and in-hand manipulation skills
allow them to string small beads and unbutton clothes. \cite{Gerber2010}

At age 4 years, a palmar tripod grasp allows for finer control of pencil
movements, and the child can copy a cross, a square, and some letters and
numerals and can draw a figure of a person (the head and a few other body
parts). Scissor skills have progressed to permit the cutting of a circle. When
a child reaches the age of 5 years, he or she can dress and undress
independently, brush the teeth well, and spread with a knife. More precise
in-hand manipulation skills enable the child to cut a square with mature
scissor movements (independent finger use) and to print his or her own name and
copy a triangle using a mature tripod pencil grasp (using the fingers to move
the pencil rather than the forearm and wrist). \cite{Gerber2010}

\subsection{Desarrollo social}

Most children are born with an inherent drive to connect with others and share
feelings, thoughts, and actions. The earliest social milestone is the bonding
of a caregiver with the infant, characterized by the caregiver’s feelings for
the child. The infant learns to discriminate his mother’s voice during the
first month after birth. He cries to express distress from hunger, fatigue, or
a wet diaper. Attachment theory suggests that as the caregiver responds to
these cries and other behaviors, the infant gains confidence in the caregiver’s
accessibility and responsiveness. This behavior system promotes the
parent–child relationship that some researchers believe facilitates parental
protection, and thus infant survival. From this relationship comes the first
measureable social milestone: the smile. \cite{Gerber2011}

Children develop their sociality according to propensities for cooperation and
from their family and community experiences. In precivilized societies, social
development occurs naturally as children learn the ways of the community
through informal apprenticeship. In civilized societies, social development
often involves “socialization,” as adults coercively teach children how to
behave. \cite{Narvaez2021}

The infant smiles at first in response to high pitched vocalizations (“baby
talk”) and a smile from his caregiver; but over time, less and less stimulation
is required. Ultimately, just seeing the caregiver elicits a smile. The infant
learns that he can manipulate the environment to satisfy personal needs by
flashing a toothless grin or, alternatively, by crying. His interactions then
begin to involve to-and-fro vocalizations by 4 months. Visual skills develop as
well, and he can recognize his caregivers by sight at 5 months. Stranger
anxiety, or the ability to distinguish between familiar and unfamiliar people,
emerges by 6 months. Whereas the 4-month-old infant smiles at any adult, the
slightly older infant cries and looks nervously between his caregiver and other
adults. \cite{Gerber2011}

Joint attention is the quintessential social milestone that develops towards
the end of the first year after birth. Joint attention is the process whereby
an infant and caregiver share an experience and recognize that the experience
is being shared. The earliest demonstration of joint attention occurs around 8
months of age, when an infant follows a caregiver’s gaze and looks in the same
direction. In a few months, the infant looks back at the caregiver as an
indication of a shared interaction. The infant consistently turns her head to
the speaker when her name is called by 10 months, further demonstrating a
connectedness with her environment. \cite{Gerber2011}

Between 12 and 14 months, children begin to point to request something
(proto-imperative pointing), and they usually integrate this pointing with eye
contact directed between the object of interest and the caregiver, sometimes
accompanied by a verbal utterance. Proto-imperative pointing then proceeds to
proto-declarative pointing by 16 months of age, characterized by the child
pointing at something merely to indicate interest. Again, the pointing is
accompanied by eye contact directed between the object and the caregiver. By 18
months, he brings objects or toys to his caregivers to show them or to share
the experience. \cite{Gerber2011}

Play skills also follow a specific developmental course. Initially, an infant
holds blocks and bangs them against each other or on the table, drops them, and
eventually throws them. Object permanence allows her to realize that the blocks
are still present, even if she cannot see them. She learns that dropping the
blocks from her highchair will cause her caregiver to pick them up and return
them to her; so she repeats this “game” over and over. As fine motor and
cognitive skills develop, she starts to use objects for more specific purposes,
such as using those blocks to build a tower. By 18 months, she engages in
simple pretend play, such as using miniature representative items in a correct
fashion. For example, she pretends to talk on a toy phone or “feeds” a doll by
using a toy spoon or bottle. \cite{Gerber2011}

After his second birthday, the child begins to play with others his own age. A
rule of thumb is that a child can play effectively only in groups of children
in the same number as his age in years. Thus, a 2-year-old can play well only
with one other child. Two-year-old play often is described as “parallel”
because a child of this age often plays next to another child but not with him.
However, the 2-year-old frequently looks at his playmate and imitates his
actions. He has not yet mastered the skill of cooperation; so aggression often
is the tool of choice to obtain a desired object. \cite{Gerber2011}

By 30 months, the child uses complex pretend play, such as using generic items
to represent other objects. A block may be used as a telephone in one scenario
or used as a bottle to feed a doll in another. The scenarios themselves also
increase in complexity, from merely feeding the doll to dressing the doll and
putting her to “sleep.” \cite{Gerber2011}

By age 3 years, a child has mastered her aggression to some extent, and she is
able to initiate a cooperative play experience with one or two peers. Most of
the time, they are able to have joint goals and take turns. She also move into
simple fantasy or imaginative play. She may pretend to be a dog or an airplane.
However, she cannot yet distinguish between what is real and what is
make-believe; so fear of imaginary things is common at this time.
\cite{Gerber2011}

Four-year-olds usually have mastered the difference between real and imaginary.
They become interested in tricking others and concerned about being tricked
themselves. They are able to play effectively with up to three other children,
although some may have a preferred friend. Imaginary scenarios increase in
complexity: a cardboard box may become a sailboat, and toilet paper rolls may
become binoculars. \cite{Gerber2011}

By age 5, children have learned many adult social skills, such as giving a
positive comment in response to another’s good fortune, apologizing for
unintentional mistakes, and relating to a group of friends. Their imaginative
play is increasingly more complex, and they love to dress up and act out their
fantasies. Kindergarten classrooms usually are well-equipped with toys that
promote this imaginative play. \cite{Gerber2011}

\subsection{Desarrollo emocional}
Coinciding with the development of social skills is a child’s emotional
development. As early as birth, all children demonstrate individual
characteristics and patterns of behavior that constitute that individual
child’s temperament. Temperament influences how an infant responds to routine
activities, such as feeding, dressing, playing, and going to sleep. There seems
to be a biologic basis to these characteristics, although how a child learns to
regulate her emotional state also depends on the interactions between child and
caregiver. \cite{Gerber2011}

Emotional development involves three specific elements: neural processes to
relay information about the environment to the brain, mental processes that
generate feelings, and motor actions that include facial expressions, speech,
and purposeful movements. The limbic system is responsible primarily for
receiving, processing, and interpreting environmental stimuli that produce
emotional responses. During development, the repertoire of specific emotions
remains constant, but the stimuli that produce them become more abstract.
\cite{Gerber2011}

Studies have demonstrated that three distinct emotions are present from birth:
anger, joy, and fear. All infants demonstrate universal facial expressions that
reveal these emotions, although they do not use these expressions
discriminately before the age of 3 months. Cognitive input is not a
requirement; anencephalic infants may show disgust with sour flavors and
pleasure with sweet flavors, just as normocephalic infants do.
\cite{Gerber2011}

Eventually, however, cognitive skills play a role as emotional expressions
become connected to specific occurrences. For example, an 8-month-old infant
can let his parents know that he is upset about being left alone in his crib or
happy about playing with a toy. Because he now has object permanence, he
demonstrates fear in new situations due to the ability to shift attention and
recognize “familiar” from “unfamiliar.” \cite{Gerber2011}

Emotional development continues as the toddler learns to identify different
emotions in other people. At 15 months, a child demonstrates empathy by looking
sad when she sees someone else cry. She also develops self-conscious emotions
(embarrassment, shame, pride) as she evaluates her own behavior in the context
of the social environment. Having once performed cute tricks on demand, she
suddenly seems embarrassed and refuses to perform when she realizes that others
are watching. She may hide behind a chair to have a bowel movement and become
upset if someone catches her in the act. \cite{Gerber2011}

As language skills develop, the child can label different emotional states in
others and even associate language with emotions and memory. For example, if he
had a tantrum when he didn’t get a toy from the store, he may have an identical
emotional outburst when he hears a verbal reminder of the situation. By age 2
years, he starts to mask emotions for social etiquette. \cite{Gerber2011}

During the preschool years, children learn more and more behavioral strategies
to manage their emotions, depending on a given situation. They begin to
understand that their expressed emotion —whether a facial, vocal, or behavioral
expression— does not necessarily need to match their subjective emotional
experience. They demonstrate an increased understanding and use of “display
rules.” These are “culturally defined rules that guide a person’s decision to
alter emotional behavior consistent with the demands of the social context.”
\cite{ZEMAN2006}

Children learn to substitute their expressions (smile and say “thank you” even
though they are disappointed in the birthday present), amplify expressions
(exaggerate a painful response to get sympathy), neutralize expression (put on
a “poker face” to hide true feelings), or minimize emotion (look mildly upset
when feeling extremely angry). By the time they enter kindergarten, children
have started to master many of the emotional nuances of social interactions.
\cite{Gerber2011}


