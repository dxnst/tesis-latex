\section{Hitos del desarrollo infantil}
\subsection{Desarrollo cognitivo}
Cognitive development is the foundation of intelligence. The dictionary defines
intelligence as the ability to learn or understand or to deal with new
situations. In reality, intelligence is a broad concept that involves multiple
factors and is incompletely understood. The best efforts to quantify this
concept come through use of standardized intelligence tests that attempt to
measure multiple areas, such as problem-solving, language, attention, memory,
and information processing. \cite{Willks2010}

Cognitive development occurs alongside the growth and maturation of a child’s
brain, as the child learns to explore, reason, solve problems, and think. This
development happens over a lifetime, as neurons develop and neural connections
form and are pruned through experiences. Scientific research continues to find
that early responsive relationships and safe, nurturing environments protect
brain development and allow for healthy growth in children. \cite{Crotty2023}

In utero, the brain is developing rapidly, with both genetic and uterine
environmental factors contributing to neurologic growth. After birth, cognitive
development begins through sensory and motor inputs and caregiver-infant
interactions. At approximately 1 to 2 months of age, infants follow a
caregiver’s face. At approximately 3 months, they may follow an object in a
circle, regard toys, and reach for faces. Six-month-olds may turn their head to
look for a dropped object and interact with toys. Nine-month-olds will search
for objects under a cloth or play peek-a-boo, demonstrating an understanding of
object permanence. Twelve-month-olds enter a new world of exploration with
first words, pointing for requests, and finding toys in containers. The first
year of cognitive development is incredible, as infants are continually
learning new communication, problem-solving, and adaptive skills.
\cite{Crotty2023}

During the toddler years, children continue to experience an explosion of
cognitive skill growth and learning. As infants and toddlers become more mobile
and curious, they are more able to explore and interact with their environment,
supporting growth in problem-solving skills. Two-year-olds can learn to open
doors, pull off clothing, and make requests. Toddlers are constantly busy,
learning from interactions and experiences. Interacting with nature, toys,
books, and peers is optimal for neural connections to be formed and pruned;
screen time at this age, which limits the physical exploration of the
environment and playful interactions with others, may hamper cognitive
development. In addition, both screen time by the child and background
television exposure seem to have negative effects on behavior in young
children. Three-year-old children may want to help with household tasks, begin
completing puzzles, and increasingly play interactively with other children.
\cite{Crotty2023}

Preschool years are critical to set the foundation for a lifetime of learning
and exploring. Children who come to school healthy with regular pediatric
health supervision visits, proper sleep hygiene, limited exposure to screen
time, and early exposure to language, singing, games, colors, and shapes are
better primed for learning. School readiness has both pre-academic and
socioemotional components and can have long-term effects on a child’s school
success, health, and quality of life. School readiness depends on both the
child and the caregiver being ready for school, taking into account caregiver
and child health and mental health and child cognitive development. A safe and
nurturing environment can support the development of skills needed for
participating in school. Preschool is the age of learning independence,
including dressing, toilet training, telling stories, interacting with art and
music, and playing with friends. When children are exposed to singing, stories,
rhymes, counting, art, and outdoor play in the preschool years, they will come
to school more prepared for learning. Early skills are foundational to success
in the early school years. \cite{Crotty2023}

\subsection{Resumen de los hitos del desarrollo}

\section{Teorías del desarrollo y la cognición}
\subsection{Psychoanalytic Theories}
\subsubsection{Sigmund Freud: Psychosexual Development}
Sigmund Freud (1856–1939) posited that behavior was motivated by a life force
—the libido— that at different developmental stages was primarily expressed in
different parts of the body. In the first year of life the child focuses on
oral sensations such as sucking, focusing primarily on relations with the
mother. Freud called the unbridled quest for need satisfaction that begins at
birth and continues through life the “id.” \cite{Feldman3}

Libidinal energy then shifts to anal needs (roughly age 1–3 years). These
sensations metaphorically relate to issues of control: of holding on and
resistance to letting go. Toilet training occurs during this period. To cope
with the pressures to regulate these needs, the child develops the “ego.” The
ego provides the means of self-control to mediate between the child’s id and
the demands of external reality. When the conflict between the id and the ego
cannot be resolved, the ego resorts to solutions acceptable to the id, termed
“defense mechanisms.” \cite{Feldman3}

Children then move to the phallic stage (ages 3–5 years). The child is driven
by sexual attraction to the parent of the opposite sex and wishes to harm the
parent of the same sex. There are strong strictures against fulfilling these
desires. This tension is resolved by identifying with the parent of the same
sex. This resolution is achieved by developing the “superego,” which serves
both as a conscience and a codebook of socially acceptable behavior.
Difficulties in the phallic stage are thought to influence the lifelong
expression of sexuality and the individual’s moral standards. \cite{Feldman3}

Next the child enters the latency period, in which Freud conceptualized minimal
tension. That stage ends when the hormonal changes of puberty manifest
themselves and start the genital stage (adolescence). Many of the battles of
childhood arise again and need to be refought. The results shape the pattern of
subsequent development and the strength of the ego and the superego.
\cite{Feldman3}

Freud’s theory was based on interviewing his adult patients about their dreams
and their childhood memories, not based on observations of children. Freud’s
emphasis was on the development of boys and on heterosexual identity and
relationships. Freud’s depiction, particularly of the phallic stage, has been
very controversial as society has changed its norms for sex role and sexuality.
\cite{Feldman3} Although Freudian ideas have been challenged, they opened the
door to subsequent theories of development. \cite{Nelson19}

\subsubsection{Erik Erikson: Psychosocial Development}
Erikson recast Freud’s stages in terms of the emerging personality. The child’s
sense of basic trust develops through the successful negotiation of infantile
needs. As children progress through these psychosocial stages, different issues
become salient. It is predictable that a toddler will be preoccupied with
establishing a sense of autonomy, whereas a late adolescent may be more focused
on establishing meaningful relationships and an occupational identity. Erikson
recognized that these stages arise in the context of Western European societal
expectations; in other cultures, the salient issues may be quite different.
\cite{Nelson19}. Erikson described the life span with eight specific stages,
each a dynamic tension between two opposing forces:

The first stage (infancy) represents the tension between basic trust and
mistrust. It embeds Freudian analyses of the oral stage in a broader interplay
between parent and child. Reliable, sensitive child rearing builds the child’s
sense of trust in the parent, whereas negligent care builds a sense of
mistrust. These interactions lead to the individual’s proclivity to give or not
to give to others. Additionally, the child must find a balance between safety
and danger. The outcome of judging situations to be safe rather than dangerous
produces a sense of hope, which is pervasive throughout life. \cite{Feldman3}

The second stage (age 2–3 years) represents the tension between autonomy and
shame and doubt. Children’s growing language and muscular control provide
opportunities both for independence and for going beyond permissible
boundaries. Parents place limits on their children, and children must learn to
inhibit urges and practice self-control without feel- ing shame or losing
self-esteem when they fail. Arriving at the delicate balance between control of
the self and control by others generalizes to the child’s attitudes toward law
and order. \cite{Feldman3}

The third stage (age 4–5 years) represents the tension between initiative and
guilt. Like Freud, Erikson characterizes the third stage in terms of the
acquisition of sex role and a functioning superego; however, Erikson notes that
emerging phallic concerns create in boys the male modality of intrusion and in
girls the female modality of receptivity. These tendencies prompt the child to
search for the gender-appropriate models and identify with them. Males thrust
themselves into social interactions, whereas females are receptive to others’
wishes. The phallic analogy of intrusion is extended to ego skills that enable
children to create their own projects and purposes. These are fostered by the
capacity for imaginative play. But imaginative activities and children’s plans
can lead into taboo territory. Children must learn to discern for themselves,
not just from parental directives, when they have crossed the line into
unacceptable thoughts, feelings, and actions. This tension is the origin of the
superego and guilt. Attempts to find the appropriate balance of personal
initiatives with moral strictures shape the child’s sense of purpose.
\cite{Feldman3}

Erikson was a pioneer in describing development throughout the life span. For
Erikson the healthy, mature adult was productive at work and was in a mutually
satisfying relationship. The latter stages of development were steps toward
achieving these endpoints. \cite{Feldman3}

\subsection{Cognitive Theories}
Cognitive development is best understood through the work of Piaget. A central
tenet of Piaget’s work is that cognition changes in quality, not just quantity.
Piaget described how children actively construct knowledge for themselves
through the linked processes of assimilation (taking in new experiences
according to existing schemata) and accommodation (creating new patterns of
understanding to adapt to new information). In this way, children are
continually and actively reorganizing cognitive processes. \cite{Nelson19}

\subsubsection{Jean Piaget: Cognitive Development}
Jean Piaget (1896–1980) focused his theory of development on how the child
develops a logical and scientific framework for understanding the physical
world. He posited that children’s understanding goes through a series of
age-linked qualitative changes or stages. Each stage is a filter that selects
and organizes what the child perceives and understands. Each stage is a
framework that provides the building blocks for the next, thus there is a
definite order in which understanding emerges. Development occurs when children
discover a discrepancy between their current understanding of reality
(assimilation) and the features of the world that don’t mesh with that
understanding (accommodation). To resolve the mismatch, children modify their
framework. The novel framework may open up new challenges that are more
complicated and far reaching. \cite{Feldman3}

Piaget proposed the most well-known theory of cognitive development. In this
theory, cognitive development unfolds in four stages from infancy to
adolescence. Each stage is qualitatively distinct, and the type of thinking the
child uses at a stage is consistent across areas of mental functioning. The
stages build on each other, and the shift from stage to stage represents
increasingly adaptive ways of thinking. Piaget viewed cognitive development
as an inherent property of human biology, and therefore he considered the
stages universal. His is a constructivist view in that knowledge develops
through children’s activities and efforts to make sense of their experiences.
Because the timing of experience can vary across children, Piaget assigned
approximate ages for the stages. \cite{Gauvain2022}

Piaget described two processes that regulate cognitive development:
organization and adaptation. Organization refers to the sequential structure of
mental development from a simple to a more complex system. Adaptation pertains
to how developing knowledge matches the environment, and it includes two
complementary functions. In assimilation, new information is added to existing
knowledge, and in accommodation, existing knowledge is modified to include new
information. The purpose of adaptation is to improve alignment between the
individual’s thinking and the environment, which Piaget called equilibration.
\cite{Gauvain2022}

With development, children’s thinking changes from a focus on immediate sensory
and motor experiences and simple ways of understanding and engaging with the
world to more complex and abstract ways of thinking. The four stages focus
primarily on logical reasoning; they are the sensorimotor period (0–2 years of
age), preoperations (2–6 years of age), concrete operations (6–11 years of
age), and formal operations (over 11 years of age). \cite{Gauvain2022}

\begin{table}[htbp]
\caption{Characteristics of Major Stages in Piaget’s Theory}
\label{tab:piaget-stages}
\resizebox{\textwidth}{!}{%
\begin{tabular}{ll}
\hline
Stage and Approximate Age Range & Major Characteristics \\ \hline
Sensorimotor: birth to 2 years & \begin{tabular}[c]{@{}l@{}}Intelligence is limited to the infant’s own actions on the\\ environment. Cognition progresses from the exercise of\\ reflexes (for example, sucking, visual orienting) to the\\ beginning of symbolic functioning.\end{tabular} \\
Preoperations: 2 to 7 years & \begin{tabular}[c]{@{}l@{}}Intelligence is symbolic, expressed via language, imagery,\\ and other modes, permitting children to mentally\\ represent and compare objects out of immediate\\ perception. Thought is intuitive rather than logical and is\\ egocentric, in that children have a difficult time taking the\\ perspective of another.\end{tabular} \\
Concrete operations: 7 to 11 years & \begin{tabular}[c]{@{}l@{}}Intelligence is symbolic and logical. (For example, if A is\\ greater than B and B is greater than C, then A must be\\ greater than C.) Thought is less egocentric. Children’s\\ thinking is limited to concrete phenomena and their own\\ past experiences; that is, thinking is not abstract\end{tabular} \\
Formal operations: 11 to 16 years & \begin{tabular}[c]{@{}l@{}}Children are able to make and test hypotheses; possibility\\ dominates reality. Children are able to introspect about\\ their own thought processes and, generally, can think\\ abstractly.\end{tabular} \hline \hline
\footnotesize Fuente: Bjorklund \cite{Bjorklund2011-aa}
\end{tabular}%
}
\end{table}

\paragraph{Sensorimotor Stage}
During the sensorimotor stage, which unfolds over the first 24 months of life,
a child’s body is the source of experiences. The child is born with a set of
rigid motor activities (reflexes), such as sucking or grasping. Initially the
reflexive acts are performed separately and directed toward specific objects.
Gradually the child begins to adapt these reflexes to different objects, such
as sucking differently on a nipple than on one’s fingers. Later the infant
combines these reflexes as in grasping the object that is sucked. This
experimentation leads to a better understanding of the external objects to
which the child is trying to adapt. The ability to use more than one action
simultaneously enables the child to construct sequences that facilitate the
creation of sequential goal-directed behavior. In the second year of life these
reflexes become internalized. The child’s ability to represent leads to
language learning since language is representational. The first words, however,
may be idiosyncratic. One child called dogs “voo-voos,” based on the barking
sound of dogs. Representational thought further enables the toddler to infer
that objects that disappear from view and are outside the range of action
nevertheless exist. By the end of the sensorimotor period the child is able to
conceptualize a world that exists independent of one’s actions upon it (object
permanence), and consequently the child can play hide and seek.
Representational thought also enables the child to watch someone perform a
novel behavior and then carry out the same behavior later (deferred imitation).
\cite{Feldman3}

\paragraph{Preoperational Stage}
By the end of the sensorimotor period toddlers are able to keep in mind objects
and events that they cannot see, but those early representational skills are
limited and continue to develop during the preoperational period, which lasts
up to 6 to 7 years of age. During this period the child slowly starts to use
words that have shared conventional meaning for all users of the language.
Private speech, based on words that remind the child of objects, is displaced
by arbitrary words such as “dog.” But the child struggles with understanding
that different people who observe the same object from different perspectives
gain different information.
Preoperational children think that their own point of view is shared
universally. They also tend to center their attention on one salient aspect of
an object or event and ignore less salient aspects. When milk is poured from a
wide cup into a tall, thin glass, the child does not recognize that although
the level of milk has changed, the amount of milk has not changed but is merely
redistributed. The child focuses on the initial and final states, not on the
transformation. Likewise, the preoperational child judges the moral value of an
act (its being good or bad) by the amount of damage, not the intentions of the
perpetrator. \cite{Feldman3}

\paragraph{Concrete Operations Stage}
Concrete operations, which arise between 7 and 11 years of age, enable
overcoming egocentrism and failures of conservation by constructing logical
frameworks or sets of rules that can be coordinated with one another. This kind
of understanding is fostered by formal education. The child understands that
liquid poured from one container into the next can be poured back to its exact
original state; that is, change in the level of the liquid is reversible.
Reversible relations underlie arithmetic (e.g., addition and subtraction) and
categorization (class inclusion and class subdivision) and can be used to
calculate changes in perspective. The deployment of these relations or
operational thinking enables the child to develop a more accurate understanding
of space, time, and morality. For example, the child now makes moral
evaluations by examining the intentions of the actor. \cite{Feldman3}

\paragraph{Formal Operations Stage}
Operations are lifted to a higher, more abstract and integrated level during
the final stage of cognitive development, which starts at 11 to 12 years of
age. The child in the concrete operations stage deals with problems through
induction, making generalizations based on a few encountered instances.
However, the teenager in the formal operational stage can make hypotheses based
on imagined possibilities and can test their current and future validity
(deduction) using propositional logic. \cite{Feldman3}

\subsubsection{Lawrence Kohlberg: Moral Development}
Lawrence Kohlberg’s (1927–1987) theory of moral reasoning is one of the best
known applications of Piaget’s theory. Kohlberg asserted that there was an
invariant and unidirectional sequence of moral developmental stages reflecting
changes in how the individual conceptualizes the world. \cite{Feldman3}

Kohlberg developed a theory of moral development in six stages, from early
childhood through adulthood. Preschoolers’ earliest sense of right and wrong is
egocentric, motivated by externally applied controls. In later stages, children
perceive equality, fairness, and reciprocity in their understanding of
interpersonal interactions through perspective taking. Most youth will reach
stage 4, conventional morality, by mid-to late adolescence. The basic theory
has been modified to distinguish morality from social conventions. Whereas
moral thinking considers interpersonal interactions, justice, and human
welfare, social conventions are the agreed-on standards of behavior particular
to a social or cultural group. Within each stage of development, children are
guided by the basic precepts of moral behavior, but they also may take into
account local standards, such as dress code, classroom behavior, and dating
expectations. There is a broader understanding of moral development of eve
young infants and children theorizing an innate capacity to relate to others.
Moral development can be found in very young infants, toddlers, and
preschoolers who have a concept of self in relation to others, empathy and
caring for others, and may incorporate their cultural context in a way that
influences how and when moral development occurs. \cite{Nelson19}

\subsection{Sociocultural theories}
\subsubsection{Lev Vygotsky: Environmental Influences on Language and Thought}
Like Piaget, Lev Vygotsky (1896–1934) was interested in the origin of knowledge
and of reasoning skills, and like Bowlby he was interested in the effects of
human relationships on development. From Vygotsky’s perspective all aspects of
human development are the result of interactions with more experienced others.
Those interactions reflect cultural practices. In some cultures very little
speech is directed toward young children, whereas in other cultures children
are expected to take part in a conversation. \cite{Feldman3}

Learning through interaction with more experienced others starts in infancy and
continues throughout development. The more experienced other helps a child to
construct memories, solve a problem, notice aspects of the environment, or
elaborate on verbal communication. To be effective, however, those acts of
collaborative exchanges, which Vygotsky names “scaffolding,” must be sensitive
to what the child would find useful, given the child’s abilities and interests.
They must be in what Vygotsky termed “the zone of proximal development.” The
input contributes to the child’s understanding only if the child can absorb the
information and incorporate it into the repertoire of knowledge.
\cite{Feldman3}

Another important focus of Vygotsky’s theory pertains to the role of language
in shaping the child’s thinking. While thought and language emerge as
independent human abilities, language, which expresses thoughts, comes to
regulate first through its external expression but later through its becoming
internalized and expressed in the child’s mind. \cite{Feldman3}

\subsubsection{Reuven Feuerstein: Social Mediation of Cognition}
Reuven Feuerstein’s (1921–2014) theory and applied work focus on the
malleability of children’s cognition and intelligence at every developmental
stage. Feuerstein theorized that cognitive development is the product of two
modalities of interaction between the organism and the environment. The first
is the individual’s direct exposure to the environment. The second modality is
mediated learning experience, which refers to how environmental stimuli are
transformed by a “mediating” agent—usually a parent, a sibling, or a caregiver.
\cite{Feldman3}

Feuerstein distinguishes between two groups of determinants of differential
cognitive development. The first group of determinants includes genetic
factors, the level of envi- ronmental stimulation, emotional relationships
between child and the mediators, and socioeconomic status. Under unfavorable
conditions these determinants interfere with cognitive development. The second
group of determinants consists of either lack of or reduced exposure to 
mediated learning experience. Such reduction interferes with cognitive
development because the guidance that mediating agents provide regarding the
relative importance of sources of information, the classification of stimuli,
or their organization is lost. \cite{Feldman3}

\subsubsection{Urie Bronfenbrenner: Direct and Indirect Environmental Influences}
Urie Bronfenbrenner (1917–2005) expanded on the description of the environment
that the growing child experiences both directly and indirectly. He postulated
five levels of environmental influences. The most immediate is the microsystem,
consisting of social environments that the child experiences directly, such as
the family, peers, and school. The microsystem is embedded in the mesosystem,
the interlinked system of microsystems in which a person participates
(e.g., the parent-teacher association, which links the family and school). The
next level is the exosystem, which includes neighbors, social and mass media,
social services, industry, and local politics. These influence the people who
influence the child. Further out is the macrosystem, the attitudes and beliefs
of the culture. All these systems are embedded in a historical context called
the chronosystem. These levels of the environment form a network of
connections, and the individual is at its center potentially influencing
children as they actively seek to adapt to their world. \cite{Feldman3}

\subsubsection{Ann Masten: Resilience in Adverse Environmental Circumstances}
Recent ideas about human positive adaptation to adversity explicate the
processes and developmental consequences of the intricate interdependence of
individual characteristics and environmental conditions. Psychologists have
been intrigued that some children who grow under adverse conditions manage to
thrive. It was initially believed that these children were invulnerable and
that understanding their success will help with interventions for other
children who were adversely affected by their challenging circumstances.
However, accumulating research by Ann Masten (1951-) revealed that resilience
only partly resides in the individual (2001). One cannot understand resilience
without considering the characteristics of both the individual and the
circumstances in which the individual develops. Consequently, understanding
resilience requires examining both the individual’s response to adversity and
the nature of the adversity itself. \cite{Feldman3}

\subsection{Dynamic Systems Theory}
One of the newest approaches to development is the dynamic systems theory. The
processes that may account for behavioral change (e.g., genetic potential,
neurologic processes, physical characteristics, fam- ily structure, personal
goals and motives) are intertwined, not independent causal factors. Development
is the outcome of interaction of processes at many levels and many systems.
Development is shaped by forces within and outside the person, merging and
influencing each other to produce new capacities and behaviors. Development
emerges as a result of moment-by-moment actions at many levels at once and does
not follow a set universal plan. \cite{Newman2020}

The core of the theory rests on the definition of systems. All systems
(biologic and social) are composed of interdependent elements that share
interrelated functions, boundaries, goals, and identity. A dynamic system
continuously changes to carry out its functions in a way that maintains the
smooth interaction among its components, thereby preserving equi- librium. The
elements influence each other and change one another over time. Understanding
development requires considering moment-by-moment events and their interplay
with changing individual characteristics. It requires tracing a pathway from
one point in time to a later point when a new, more mature behavior emerges.
This requires a detailed analysis of multiple aspects of behavior and an
investigation of the process of reorganization and growth. The theory helps
specify the properties of systems that are in flux (open systems) and how
feedback acts to regulate change. \cite{Feldman3}

The word dynamic is used to underscore the constant interaction and mutual
influence of the elements of the system. A key component is self-organization,
the idea that development is produced through the interactions of the various
elements of the system. The system includes the biological properties of the
organism that support development, actions of the organism in using these
properties, and input from the environment. Together, these elements produce a
set of behaviors that lead to cognitive change. \cite{Gauvain2022}
