\documentclass[11pt,letterpaper]{report}
\usepackage[margin=1in]{geometry}
\usepackage[utf8]{inputenc}
\usepackage[T1]{fontenc}
\usepackage{helvet}
\renewcommand{\familydefault}{\sfdefault}
\linespread{1.5}
\usepackage[spanish,es-nodecimaldot]{babel}
\usepackage{titlesec}
\usepackage[hidelinks]{hyperref}
\usepackage[skip=0pt plus8pt, indent=1.25cm]{parskip}

\usepackage{pdfpages}
% Formatos de títulos
\titleformat{\chapter}[block]{\normalfont\fontsize{12pt}{12pt}\bfseries}{SECCIÓN \ \thechapter.}{0.6em}{\centering\MakeUppercase}
\titlespacing*{\chapter}{0em}{*3}{*2}
\titleformat{\section}[block]{\normalfont\fontsize{12pt}{12pt}\bfseries}{\thesection}{0.6em}{}
\titlespacing*{\section}{0em}{*2}{*1.2}
\titleformat{\subsection}[block]{\normalfont\bfseries\itshape}{\thesubsection}{0.6em}{}
\titlespacing*{\subsection}{4em}{*1}{*0.6}

% Formato de citación
\usepackage[backend=biber, style=vancouver]{biblatex}
\addbibresource{mydocument.bib}

% Formato de numeración
\renewcommand*{\theenumi}{\thesection.\arabic{enumi}}
\renewcommand*{\theenumii}{\theenumi.\arabic{enumii}}

% Cosas generales
\title{Factores de riesgo en el neurodesarrollo infantil}
\author{Soto Consuegra, Josué Daniel \and López Castillo, Sarah Ivón \and
Ixquiac Vásquez, Etelvina Del Rosario \and Guzmán Pérez, Mariana Del Rosario
\and Mazariegos Manrique, Sonia María}
\newcommand{\tiempito}{marzo a junio de 2025}
\newcommand{\muestradeseada}{1,701}
\newcommand{\asq}{“Cuestionario Edades y Etapas 3”}

\begin{document}
%	\chapter*{Dedicatoria}
%	\chapter*{Agradecimiento}
%	\chapter*{Resumen}
	\tableofcontents
%	\chapter{Introducción}
	\chapter{Planteamiento del problema}
\section{Descripción del problema}
\section{Delimitación del problema}
	\begin{enumerate}
		\item Ámbito geográfico:
		\item Ámbito institucional:
		\item Ámbito poblacional:
		\item Ámbito temporal:
		\item Ámbito temático:
	\end{enumerate}
\section{Preguntas de investigación}

	\chapter{Objetivos}
\section{Objetivo general}
	\begin{enumerate}
		\item Establecer la asociación entre factores sociodemográficos,
		económicos, familiares y médicos con el riesgo en el neurodesarrollo en
		niños menores de 5 años que asisten a servicios de atención primaria en
		el distrito de Quetzaltenango, mediante evaluaciones con el \asq\
		durante 2025.
	\end{enumerate}
\section{Objetivos específicos}
	\begin{enumerate}
		\item Clasificar los resultados del \asq\ según grupos de edad para
		detectar patrones específicos de riesgo en los dominios del
		neurodesarrollo.
		
		\item Evaluar la asociación entre factores socioeconómicos,
		demográficos, ambientales y antecedentes perinatales y el riesgo de
		retraso en el neurodesarrollo utilizando el \asq.
		
		\item Analizar la relación entre acceso a servicios de atención
		primaria durante el periodo prenatal y postnatal con la presencia de
		riesgo de retraso en el neurodesarrollo.
	\end{enumerate}	

	\chapter{Justificación}
Los trastornos del desarrollo, también conocidos como retrasos del desarrollo,
constituyen un grupo heterogéneo de condiciones que afectan el aprendizaje, el
lenguaje, el comportamiento o las habilidades motoras.
\cite{cdcDevelopmentalDisability} Estos retrasos se identifican cuando un niño
no alcanza los hitos de desarrollo esperados en comparación con sus pares de la
misma población \cite{DevelopmentalSurveillance}. Por ello es importante
destacar que el retraso en el desarrollo no es un diagnóstico en sí mismo, sino
un término descriptivo utilizado en la práctica clínica para indicar un
fenotipo amplio que requiere una evaluación más detallada para determinar las
áreas específicas de desarrollo afectadas. Hay tres tipos de retraso en el
desarrollo basado en el número de dominios involucrados: 1) Retraso aislado en
el desarrollo: involucra un solo dominio; 2) Múltiples retrasos en el
desarrollo: 2 o más dominios o líneas de desarrollo afectados; y, 3) Retraso
global en el desarrollo: retraso significativo en la mayoría de los dominios de
desarrollo. \cite{Bellman2013} Aunque la etiología de la mayoría de los
retrasos en el desarrollo es idiopática, cuando se identifica, puede incluir
factores genéticos, ambientales y/o psicosociales. \cite{DevelopmentalDelay}

En Guatemala, según el informe de la línea de base de la Gran Cruzada Nacional
por la Nutrición 2021/2022 de la Secretaría de Seguridad Alimentaria y
Nutricional, solo el 1.9\% de las madres de niños entre 2 y 5 años reportaron
que sus hijos habían asistido alguna vez a un programa de primera infancia, y
apenas el 0.6\% asiste actualmente a un Centro Comunitario de Desarrollo
Infantil Temprano. Más preocupante aún, solo el 49.8\% de los niños de 24 a 59
meses se encuentran en el camino adecuado de desarrollo, salud, aprendizaje y
bienestar psicosocial. \cite{SESAN2022}

A nivel global, según un reporte de UNICEF en 2023, se estima que 250 millones
de niños menores de 5 años están en riesgo de no alcanzar su potencial de
desarrollo. Aproximadamente 200 millones de niños menores de 5 años no están
creciendo, no presentan un adecuado desarrollo global, debido a la desnutrición
en la primera infancia. Además, más de 2 de cada 5 niños entre 3 y 4 años no
reciben la estimulación temprana ni el cuidado parental adecuados. Como
resultado de estas y otras amenazas, el 29\% de los niños de 3 a 5 años no
están logrando un desarrollo apropiado. \cite{UNICEF2023}

El neurodesarrollo infantil es un proceso complejo y dinámico que sienta las
bases para el futuro cognitivo, emocional y social de los individuos. En
Quetzaltenango, Guatemala, existe una brecha significativa en la investigación
sobre los factores que influyen en el desarrollo neurológico de los niños
menores de 5 años. Esta carencia de datos locales específicos obstaculizan la
implementación de intervenciones efectivas y políticas públicas adecuadas.

Para llevar a cabo este estudio en Quetzaltenango, es necesario un equipo de 5
investigadores debido a la complejidad y el alcance de la muestra, la cual
comprende \muestradeseada\ niños. La distribución del trabajo se detalla a
continuación:

	\begin{itemize}
		\item Carga de trabajo y distribución: Cada investigador estará a cargo
		de evaluar aproximadamente 340 niños, lo cual permite una división
		equitativa para asegurar una atención detallada en cada caso. Esto es
		crucial para mantener la calidad de los datos y la consistencia en la
		recolección de información, aspecto necesario para la validez del
		estudio.
		\item Tiempo estimado de evaluación: Cada evaluación individual tomará
		alrededor de 30 a 50 minutos. Esto representa aproximadamente 1,417
		horas en total o 283 horas por investigador. La presencia de 5
		investigadores optimiza el proceso y asegura que las evaluaciones se
		realicen en el tiempo programado.
		\item Cobertura de múltiples puntos de atención: La investigación se
		llevará a cabo en tres servicios de atención primaria de
		Quetzaltenango: el Centro de Salud de Quetzaltenango, el Puesto de
		Salud de Pacajá y el Puesto de Salud de San José Chiquilajá.
		\item Atención a casos en riesgo: Los niños identificados con riesgo en
		el neurodesarrollo y sus padres o tutores recibirán plan educacional y
		material de apoyo para promover actividades de estimulación temprana en
		casa. El mismo lo llevará a cabo el investigador utilizando
		herramientas recomendadas por UNICEF.
	\end{itemize}

En conclusión, la integración de un equipo de 5 investigadores permite abordar
de manera exhaustiva y precisa los desafíos de la evaluación de neurodesarrollo
en niños menores de 5 años en Quetzaltenango. Los resultados esperados no solo
aportarán evidencia científica local, sino que también promoverán
intervenciones que puedan mejorar el desarrollo integral de los niños,
sensibilizando a las autoridades y profesionales de la salud sobre la
importancia de intervenciones tempranas y costo efectivas en el desarrollo de
la primera infancia.

	\chapter{Marco teórico}
	\chapter{Población y métodos}

\printbibliography
\end{document}
