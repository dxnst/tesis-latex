\documentclass[11pt,letterpaper]{report}
\usepackage[margin=1in]{geometry}
\usepackage[utf8]{inputenc}
\usepackage[T1]{fontenc}
\usepackage{helvet}
\renewcommand{\familydefault}{\sfdefault}
\linespread{1.5}
\usepackage[spanish,es-nodecimaldot]{babel}
\usepackage{titlesec}
\usepackage[hidelinks]{hyperref}
\usepackage[skip=0pt plus8pt, indent=1.25cm]{parskip}

\usepackage{pdfpages}
% Formatos de títulos
\titleformat{\chapter}[block]{\normalfont\fontsize{12pt}{12pt}\bfseries}{SECCIÓN \ \thechapter.}{0.6em}{\centering\MakeUppercase}
\titlespacing*{\chapter}{0em}{*3}{*2}
\titleformat{\section}[block]{\normalfont\fontsize{12pt}{12pt}\bfseries}{\thesection}{0.6em}{}
\titlespacing*{\section}{0em}{*2}{*1.2}
\titleformat{\subsection}[block]{\normalfont\bfseries\itshape}{\thesubsection}{0.6em}{}
\titlespacing*{\subsection}{4em}{*1}{*0.6}

% Formato de citación
\usepackage{csquotes}
\usepackage[backend=biber, style=vancouver]{biblatex}
\DeclareFieldFormat*{url}{\bibstring{urlfrom}: \url{#1}}
\DeclareFieldFormat{urldate}{[\bibstring{urlseen}: \space#1]}
\addbibresource{mydocument.bib}

\DefineBibliographyStrings{spanish}{
	urlfrom = {Disponible en},
	urlseen = {Accedido},
}

% Formato de numeración
\renewcommand*{\theenumi}{\thesection.\arabic{enumi}}
\renewcommand*{\theenumii}{\theenumi.\arabic{enumii}}

% Cosas generales
\title{Factores de riesgo en el neurodesarrollo infantil}
\author{Soto Consuegra, Josué Daniel \and López Castillo, Sarah Ivón \and
Ixquiac Vásquez, Etelvina Del Rosario \and Guzmán Pérez, Mariana Del Rosario
\and Mazariegos Manrique, Sonia María}
\newcommand{\tiempito}{durante mayo de 2025}
\newcommand{\muestradeseada}{1,701}
\newcommand{\asq}{“Cuestionario Edades y Etapas 3”}

\begin{document}
%	\chapter*{Dedicatoria}
%	\chapter*{Agradecimiento}
%	\chapter*{Resumen}
	\tableofcontents
%	\chapter{Introducción}
	\chapter{Planteamiento del problema}
\section{Descripción del problema}
\section{Delimitación del problema}
	\begin{enumerate}
		\item Ámbito geográfico:
		\item Ámbito institucional:
		\item Ámbito poblacional:
		\item Ámbito temporal:
		\item Ámbito temático:
	\end{enumerate}\
\section{Preguntas de investigación}

	\chapter{Objetivos}
\section{Objetivo general}
	\begin{enumerate}
		\item Establecer la asociación entre factores sociodemográficos,
		económicos, familiares y médicos con el riesgo en el neurodesarrollo en
		niños menores de 5 años que asisten a servicios de atención primaria en
		el distrito de Quetzaltenango, mediante evaluaciones con el \asq\
		durante 2025.
	\end{enumerate}
\section{Objetivos específicos}
	\begin{enumerate}
		\item Clasificar los resultados del \asq\ según grupos de edad para
		detectar patrones específicos de riesgo en los dominios del
		neurodesarrollo.
		
		\item Evaluar la asociación entre factores socioeconómicos,
		demográficos, ambientales y antecedentes perinatales y el riesgo de
		retraso en el neurodesarrollo utilizando el \asq.
		
		\item Analizar la relación entre acceso a servicios de atención
		primaria durante el periodo prenatal y postnatal con la presencia de
		riesgo de retraso en el neurodesarrollo.
	\end{enumerate}

	\chapter{Justificación}
Los trastornos del desarrollo, también conocidos como retrasos del desarrollo,
constituyen un grupo heterogéneo de condiciones que afectan el aprendizaje, el
lenguaje, el comportamiento o las habilidades motoras.
\cite{cdcDevelopmentalDisability} Estos retrasos se identifican cuando un niño
no alcanza los hitos de desarrollo esperados en comparación con sus pares de la
misma población \cite{DevelopmentalSurveillance}. Por ello es importante
destacar que el retraso en el desarrollo no es un diagnóstico en sí mismo, sino
un término descriptivo utilizado en la práctica clínica para indicar un
fenotipo amplio que requiere una evaluación más detallada para determinar las
áreas específicas de desarrollo afectadas. Hay tres tipos de retraso en el
desarrollo basado en el número de dominios involucrados: 1) Retraso aislado en
el desarrollo: involucra un solo dominio; 2) Múltiples retrasos en el
desarrollo: 2 o más dominios o líneas de desarrollo afectados; y, 3) Retraso
global en el desarrollo: retraso significativo en la mayoría de los dominios de
desarrollo. \cite{Bellman2013} Aunque la etiología de la mayoría de los
retrasos en el desarrollo es idiopática, cuando se identifica, puede incluir
factores genéticos, ambientales y/o psicosociales. \cite{DevelopmentalDelay}

En Guatemala, según el informe de la línea de base de la Gran Cruzada Nacional
por la Nutrición 2021/2022 de la Secretaría de Seguridad Alimentaria y
Nutricional, solo el 1.9\% de las madres de niños entre 2 y 5 años reportaron
que sus hijos habían asistido alguna vez a un programa de primera infancia, y
apenas el 0.6\% asiste actualmente a un Centro Comunitario de Desarrollo
Infantil Temprano. Más preocupante aún, solo el 49.8\% de los niños de 24 a 59
meses se encuentran en el camino adecuado de desarrollo, salud, aprendizaje y
bienestar psicosocial. \cite{SESAN2022}

A nivel global, según un reporte de UNICEF en 2023, se estima que 250 millones
de niños menores de 5 años están en riesgo de no alcanzar su potencial de
desarrollo. Aproximadamente 200 millones de niños menores de 5 años no están
creciendo, no presentan un adecuado desarrollo global, debido a la desnutrición
en la primera infancia. Además, más de 2 de cada 5 niños entre 3 y 4 años no
reciben la estimulación temprana ni el cuidado parental adecuados. Como
resultado de estas y otras amenazas, el 29\% de los niños de 3 a 5 años no
están logrando un desarrollo apropiado. \cite{UNICEF2023}

El neurodesarrollo infantil es un proceso complejo y dinámico que sienta las
bases para el futuro cognitivo, emocional y social de los individuos. En
Quetzaltenango, Guatemala, existe una brecha significativa en la investigación
sobre los factores que influyen en el desarrollo neurológico de los niños
menores de 5 años. Esta carencia de datos locales específicos obstaculizan la
implementación de intervenciones efectivas y políticas públicas adecuadas.

Para llevar a cabo este estudio en Quetzaltenango, es necesario un equipo de 5
investigadores debido a la complejidad y el alcance de la muestra, la cual
comprende \muestradeseada\ niños. La distribución del trabajo se detalla a
continuación:

	\begin{itemize}
		\item Carga de trabajo y distribución: Cada investigador estará a cargo
		de evaluar aproximadamente 340 niños, lo cual permite una división
		equitativa para asegurar una atención detallada en cada caso. Esto es
		crucial para mantener la calidad de los datos y la consistencia en la
		recolección de información, aspecto necesario para la validez del
		estudio.
		\item Tiempo estimado de evaluación: Cada evaluación individual tomará
		alrededor de 30 a 50 minutos. Esto representa aproximadamente 1,417
		horas en total o 283 horas por investigador. La presencia de 5
		investigadores optimiza el proceso y asegura que las evaluaciones se
		realicen en el tiempo programado.
		\item Cobertura de múltiples puntos de atención: La investigación se
		llevará a cabo en tres servicios de atención primaria de
		Quetzaltenango: el Centro de Salud de Quetzaltenango, el Puesto de
		Salud de Pacajá y el Puesto de Salud de San José Chiquilajá.
		\item Atención a casos en riesgo: Los niños identificados con riesgo en
		el neurodesarrollo y sus padres o tutores recibirán plan educacional y
		material de apoyo para promover actividades de estimulación temprana en
		casa. El mismo lo llevará a cabo el investigador utilizando
		herramientas recomendadas por UNICEF.
	\end{itemize}

En conclusión, la integración de un equipo de 5 investigadores permite abordar
de manera exhaustiva y precisa los desafíos de la evaluación de neurodesarrollo
en niños menores de 5 años en Quetzaltenango. Los resultados esperados no solo
aportarán evidencia científica local, sino que también promoverán
intervenciones que puedan mejorar el desarrollo integral de los niños,
sensibilizando a las autoridades y profesionales de la salud sobre la
importancia de intervenciones tempranas y costo efectivas en el desarrollo de
la primera infancia.

	\chapter{Marco teórico}
	\chapter{Población y métodos}
\section{Tipo y diseño de la investigación}
Estudio de enfoque cuantitativo, diseño analítico, observacional, prospectivo
de corte transversal.

\section{Unidad de análisis}
	\begin{enumerate}
		\item Unidad primaria de muestreo: Servicios de atención primaria en
		salud de la ciudad de Quetzaltenango, en específico el Puesto de
		Salud de San José Chiquilajá, Puesto de Salud de Pacajá y el Centro de
		Salud de Quetzaltenango.
		\item Unidad de análisis: Información sobre aspectos sociodemográficos,
		económicos, familiares, perinatales, nutricionales, médicos, de
		interacción y estimulación de los niños y su evaluación de riesgo de
		acuerdo a los dominios del desarrollo de comunicación, área motora
		gruesa y fina, resolución de problemas y área socio-individual.
		\item Unidad de información: Madres o encargados y niños que acudan a
		servicios de atención primaria de la ciudad de Quetzaltenango.
	\end{enumerate}

\section{Población y muestra}
	\begin{enumerate}
		\item Población o universo: Niños menores de 5 años en el área de salud
		del distrito de Quetzaltenango.
		\item Marco muestral: Niños menores de 5 años que acuden a servicios de
		atención primaria en el Puesto de Salud de San José Chiquilajá, Puesto
		de Salud de Pacajá y el Centro de Salud de Quetzaltenango. %\tiempito. 
		\item Muestra: \muestradeseada\ niños menores de 5 años que acudan a
		servicios de atención primaria seleccionados en Quetzaltenango.
	\end{enumerate}

El tipo de muestreo será no probabilístico por conveniencia, incluyendo a todos
los niños que cumplan con los criterios de inclusión y asistan a los servicios
de atención primaria participantes, hasta alcanzar el tamaño
de muestra deseado de \muestradeseada\ niños.

\section{Selección de los sujetos a estudio}
	\begin{enumerate}
		\item Criterios de inclusión:
			\begin{itemize}
				\item Niños de 0 a 59 meses de edad que acuden a servicios de
				atención primaria para controles de crecimiento y desarrollo,
				vacunación o consulta médica.
				\item Padres o cuidadores que acepten participar en el estudio
				y firmen el consentimiento informado.
			\end{itemize}
		\item Criterios de exclusión:
			\begin{itemize}
				\item Niños con diagnóstico previo de trastornos del
				neurodesarrollo o discapacidad intelectual
				\item Padres o cuidadores que no acepten participar en el
				estudio o se retiren durante el proceso.
			\end{itemize}
	\end{enumerate}

\section{Definición y operacionalización de variables}
\includepdf[scale=0.90,landscape=true,pages=-,pagecommand={\thispagestyle{plain}}]{OpVariables.pdf}

\section{Hipótesis}
	\begin{enumerate}
		\item Hipótesis nula (H0): No existe una asociación significativa entre
		factores sociodemográficos, condiciones económicas, interacción
		familiar, exposición a dispositivos electrónicos, antecedentes médicos
		perinatales y postnatales, y el riesgo en el neurodesarrollo de niños
		menores de 5 años en servicios de atención primaria de Quetzaltenango.
		\item Hipótesis alternativa (H1): Existe una asociación significativa
		entre factores sociodemográficos, condiciones económicas, interacción
		familiar, exposición a dispositivos electrónicos, antecedentes médicos
		perinatales y postnatales, y el riesgo en el neurodesarrollo de niños
		menores de 5 años en servicios de atención primaria de Quetzaltenango.
	\end{enumerate}

\section{Técnicas de recolección de información e instrumentos de medición}
	\begin{enumerate}
		\item Técnicas de recolección de información: Para llevar a cabo este
		estudio de cohorte prospectivo, se implementarán las siguientes fases:
	\begin{enumerate}
		\item Fase preliminar (Febrero de 2025):
		Se obtuvieron los permisos correspondientes a las autoridades de salud
		del departamento de Quetzaltenango para acceder a los servicios de
		atención primaria seleccionados. Se determinaron estrategias para
		garantizar la uniformidad en la recolección de los datos entre los
		investigadores.
		\item Fase de recolección de datos:
		Se identificarán y reclutarán niños menores de 5 años que cumplan con
		los criterios de inclusión en los servicios de atención primaria
		participantes. Tras obtener el consentimiento informado de los padres o
		tutores, se realizará:
			\begin{itemize}
			\item Evaluación basal del neurodesarrollo mediante la aplicación
			del \asq, seleccionando la versión específica según la edad del
			niño.
			\item Aplicación de un cuestionario estructurado para recolectar
			información sobre factores potencialmente asociados al
			neurodesarrollo.
			\end{itemize}
		\item Fase de clasificación y análisis:
		Los resultados de cada niño serán evaluados conforme al
		puntaje obtenido en el \asq\
		y clasificados en tres categorías:
			\begin{itemize}
			\item Desarrollo típico: puntaje en el área blanca, indicativo de
			un desarrollo acorde a su edad.
			\item Requiere monitoreo: puntaje en el área gris, señalando
			habilidades ligeramente por debajo del promedio.
			\item Retraso en el desarrollo: puntaje en el área negra,
			sugiriendo la necesidad de intervención especializada.
			\end{itemize}
		Se analizarán las asociaciones entre los factores de exposición
		identificados y los resultados de neurodesarrollo en la evaluación.
	\end{enumerate}
	\item Instrumentos de recolección de información
	Para este estudio de cohorte prospectivo, se emplearán los siguientes
	instrumentos:
		\begin{itemize}
		\item \asq: Adaptado al idioma español y ajustado por edad. Esta
		herramienta validada de tamizaje del desarrollo identifica riesgos de
		problemas de neurodesarrollo en niños de 2 a 66 meses. Será aplicado
		por los investigadores con información proporcionada por los padres o
		tutores y mediante observación directa de actividades específicas.
		El \asq\ evalúa cinco áreas del desarrollo: comunicación, motricidad
		gruesa, motricidad fina, resolución de problemas, habilidades
		socioindividuales.

		\item Cuestionario de factores de exposición: Instrumento estructurado
		diseñado específicamente para este estudio que recopilará información
		sobre:
				\begin{itemize}
					\item Variables sociodemográficas (edad, sexo, etnia, nivel
					educativo de los padres)
					\item Variables económicas (empleo de los padres, acceso a
					seguridad social)
					\item Variables de interacción familiar (tiempo de juego,
					disponibilidad de juguetes)
					\item Variables médicas (prematuridad, peso al nacer, tipo
					de parto, lactancia, estado nutricional, etc.)
				\end{itemize}
		\end{itemize}
\end{enumerate}

\section{Plan de análisis de datos}
\begin{enumerate}
	\item Preparación de los datos: Los datos en formato físico serán
	digitados para su uso en el software estadístico Rstudio y python con los
	paquetes numpy y pandas. Se realizará una limpieza de los datos para
	identificar y corregir posibles errores de entrada. Los puntajes obtenidos
	en cada área del desarrollo del \asq\ se convertirán a valores
	estadísticos. 
	
	\item Análisis descriptivo de datos de la cohorte completa: Se calcularán
	frecuencias y porcentajes de los diferentes factores de riesgo presentes en
	la población a estudiar. Se calcularán medidas de tendencia central como
	media, mediana, y desviación estándar de los puntajes del neurodesarrollo.

	\item Análisis comparativo de los resultados del \asq\ de la cohorte
	completa utilizando las siguientes herramientas estadísticas:
		\begin{itemize}
		\item Chi-cuadrado: para determinar si hay asociación significativa
		entre las variables categóricas y riesgo del retraso en el
		neurodesarrollo, se utilizará para evaluar factores de riesgo
		individuales y comparar con desarrollo normal versus desarrollo en
		riesgo.
		\item Análisis de variancia (ANOVA): para comparar medias de puntajes
		del neurodesarrollo en más de dos grupos diferentes de una misma
		categoría y determinar su variación, por ejemplo para evaluar el riesgo
		del neurodesarrollo en valores Z y medidas de tendencia central con
		el grado de escolaridad de los padres de los niños: ninguna, primaria,
		básico, diversificado, universitario.
		\end{itemize}

	\item Análisis de asociación de los resultados del \asq\ del grupo
	estudiado utilizando:
		\begin{itemize}
		\item Odds ratio (OR): para comparar las probabilidades de que se
		presente riesgo en el neurodesarrollo entre dos grupos diferentes.
		Por ejemplo para comparar si los niños con padres que tienen un trabajo
		formal o informal tienen mayor probabilidad o no, de presentar riesgo
		en el neurodesarrollo.
		\end{itemize}
	\item Presentación de resultados: se elaborarán tablas y gráficos
	apropiados con intervalos utilizando el software Rstudio y paquetes de
	CRAN como ggplot2 para análisis y creación de datos informativos.
\end{enumerate}

\section{Principios éticos en la investigación}
Esta investigación se adherirá a los principios éticos clave, tales como:
\begin{itemize}
	\item Consentimiento informado: explicando claramente los objetivos del
	estudio a los padres o tutores y obteniendo su autorización.

	\item Confidencialidad: los datos se mantendrán anónimos y se utilizarán 
	exclusivamente para fines de investigación.

	\item Beneficencia y no maleficencia: buscando maximizar beneficios
	potenciales sin causar daños a los participantes.

	\item El \asq\ es una herramienta validada, respaldada por evidencia
	científica y recomendada por instituciones como UNICEF para su uso en
	evaluación del neurodesarrollo infantil en servicios de atención de
	salud. \cite{UNICEFrespaldo}
\end{itemize}

\printbibliography
\end{document}
