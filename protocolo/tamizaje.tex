\section{Evaluación del neurodesarrollo infantil}
\subsection{Cuestionarios Edades y Etapas 3}
Los ``Cuestionarios Edades y Etapas 3'' son una serie de herramientas de
cribado que identifican de forma precisa el riesgo de los niños de problemas
del neurodesarrollo, son un conjunto de 21 cuestionarios diferentes para
distintos rangos de edad desde 1 mes hasta los 5 años y 6 meses de vida. Son
cuestionarios que completan los padres de los niños en conjunto con un
proveedor de salud. \cite{Singh2017}

Son un instrumento de detección del desarrollo reportado por los padres, cada
uno con 30 elementos en cinco áreas del neurodesarrollo: personal social,
motricidad gruesa, motricidad fina, resolución de problemas y comunicación para
niños de 2 a 66 meses. En la mayoría de los casos, estos cuestionarios
identifican con precisión a los niños pequeños que necesitan una evaluación
adicional para determinar si son elegibles para los servicios de intervención
temprana. \cite{Singh2017}

Psychometric parameters of the ASQ have been examined based on completion of
18,000 respondents. \cite{squires2009ages} Evidence shows that the ASQ is an
accurate, cost-effective, parent-friendly instrument for screening and
monitoring of preschool children. In addition, it is recommended for early
detection of autism by the Joint Committee on Screening and Diagnosis of Autism
as well as for general developmental follow-up and screening and developmental
surveillance in office settings. Furthermore, research shows that the ASQ has
been successfully used for follow-up and assessment of premature and at-risk
infants and children in the public health, and follow-up of infants born after
assisted reproductive technologies. \cite{Chiu2010, Yu2007}

\subsubsection{Contexto histórico de los Cuestionarios Edades y Etapas 3}
El desarrollo de los ``Cuestionarios Edades y Etapas'' inició en 1980 en la
Universidad de Oregon, con la finalidad de evaluar el desarrollo neurológico de
los niños con habilidades que los padres son capaces de reconocer e identificar
en casa. En la década de 1980 a 1990 las doctoras Diane Bricker y Jane Squires
realizaron una búsqueda exhaustiva de conjuntos de habilidades fáciles de ser
elicitadas u observadas por los padres de los niños. En 1995 se publicó una
serie de 8 cuestionarios que evaluaban niños hasta los 48 meses de edad.
\cite{ASQ4decades}

En 1996 se iniciaron estudios de validez, fiabilidad y utilidad de la primera
edición del ``Cuestionario Edades y Etapas''. Para el año 1997 y 1998 se los
estudios determinan las propiedades psicométricas del instrumento de tamizaje.
\cite{ASQ4decades}

En 1999 se publica la segunda edición de los cuestionarios revisados y
extendidos para cubrir hasta la edad de 60 meses. En 2000 se inician
investigaciones a nivel internacional en Finlandia, Noruega, China, Portugal y
Brasil. En 2001 la Academia Americana de Pediatría recomienda el ``Cuestionario
Edades y Etapas'' como una herramienta con buenas propiedades psicométricas,
incluyendo sensibilidad, especificidad, validez y fiabilidad adecuadas, y
estandarizada en diferentes poblaciones. \cite{Pediatrics2001}

Para 2004 se inicia una recolección de datos para la tercera edición del
cuestionario y durante 4 años se recolectan 18,000 cuestionarios de niños de 50
estados de territorios estadounidenses. \cite{ASQ4decades}

Para 2006 la Academia Americana de Pediatría revisa su política de tamizaje del
desarrollo con un algoritmo que incluye cribado a los 9, 18 y 30 meses en
chequeos médicos de rutina y recomienda el ``Cuestionario Edades y Etapas''
como una de las principales herramientas para el tamizaje.
\cite{Pediatrics2006}
 
En 2009 la tercera edición de los cuestionarios es publicada, incluyendo nuevos
valores de estandarización, y puntajes revisados para catalogar a los niños en
áreas de riesgo o de monitoreo. Esta tercera edición se ajusta a los resultados
de 18,000 cuestionarios obtenidos de diferentes contextos socioeconómicos y
lugares en los Estados Unidos de América. \cite{ASQ4decades}

\subsubsection{Viabilidad de los Cuestionarios Edades y Etapas 3 en Guatemala}
Tiene excelentes propiedades psicométricas, fiabilidad de
evaluación-reevaluación del 92\%, sensibilidad del 87.4\% y especificidad del
95,7\%. La validez ha sido examinada en diferentes culturas y comunidades de
todo el mundo.
\cite{Vameghi2013-uo, SarmientoCampos2010, Heo2007, Saihong2010, CMehner2019}

En la aplicabilidad de los cuestionarios en países de medianos y bajos
recursos, la evidencia indica que los cuestionarios no pueden ser contestados
de forma individual por los padres de los niños, y que los resultados son más
fiables cuando los cuestionarios son contestados en conjunto con un proveedor
de salud. \cite{Manasyan2023, Colbert2021}

En los Estados Unidos de América es ampliamente utilizada en programas de
chequeo médico de rutina por recomendaciones de la Academia Americana de
Pediatría de realizar por lo menos un tamizaje del desarrollo a las edades de
9, 18 y 30 meses de edad. Otros programas como ``Early Head Start'', ``Help Me
Grow'', ``Child Find'', ``Parents as Teachers'', y programas locales de
condados realizan tamizajes y seguimientos del neurodesarrollo con el
``Cuestionario Edades y Etapas 3'' con mayor frecuencia. \cite{ASQWorld}
