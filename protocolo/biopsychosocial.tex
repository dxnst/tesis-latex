\section{Modelo biopsicosocial del desarrollo infantil}
Biology influences behavior and environment, and behavior and environment
influence biology throughout development. Children are directly and indirectly
influenced by both their proximal context (e.g., relationships with their
caregivers) and broader societal factors (e.g., neighborhood violence,
community-wide belief systems). Children’s development is the product of the
accumulation of everyday interactions and experiences as well as the broader
community and cultural context in which they are raised. While major events
(e.g., changes in family structure) and circumstances (e.g., family resources)
are important to children’s development, so too are the minute interactions
that make up day-to-day life. The multilevel and transactional influences on
children’s develop- ment have been described in two key theoretical models.
\cite{Feldman3}

Early influences, particularly those producing toxic levels of stress, affect
the individual through their impact on the body’s stress response systems,
brain development, and modification of gene expression. Epigenetic changes,
such as DNA methylation and histone acetylation, may be influenced by early
life experiences (the environment) and impact gene expression without changing
the DNA sequence. These changes can produce long-lasting effects on the health
and wellbeing of the individual and may be passed on to future generations.
\cite{Nelson19}

\subsection{Major theories}

\subsubsection{Bronfenbrenner's Ecological Systems Theory}
Urie Bronfenbrenner's ecological systems theory
proposes there are multiple levels of influence on a child’s development,
spanning from relationships with caregivers to systems such as schools and
workplaces, to events in the broader society. The microsystem
describes the direct relationships and interactions chil- dren have, such as
with caregivers, siblings, and peers. These individuals directly influence the
child by scaffolding development, providing opportunities to play and learn,
and providing emotional support to children. The microsystem also contains
structures with which the child interacts, such as school, neighborhood,
childcare settings, and family. Children both influence and are influenced by
these relationships and structures (e.g., a child’s temperament may contribute
to setting the tone of a classroom). The mesosystem describes interaction among
the structures that are in the microsystem (e.g., bidirectional influences
between neighborhoods and schools). The exosystem consists of larger social
systems that impact structures in the microsystem (e.g., community-based family
resources or parental work schedules). Children do not directly interact with
the exosystem, but they experience the impact of changes in these social
systems. The macrosystem is the outermost layer of a child’s environment and is
defined by cultural values, customs, and laws that influence the ways that the
inner layers function. The chronosystem captures the influence of time on
children’s development, reflecting both developmental processes that take place
over time and the changing influence of events (e.g., a traumatic event) based
on their duration and the developmental stage in which they occur.
\cite{Feldman3}

\subsubsection{Sameroff's Transactional Model}
Arnold Sameroff's transactional model builds on Bronfenbrenner's ideas about
the bidirectionality of effects on children’s development. It discusses the
processes that take place between parents and children in everyday interactions
and over time. This model grew out of observations that many risks, such as
premature birth or birth complications, were associated with observable
developmental problems only for some children, most often those children with
additional social risks (e.g., low socioeconomic status). In other words,
children’s environments moderate the effect of early biologic risks on
children’s development. Nature and nurture are viewed as inherently
inextricable; genes are expressed dependent on one’s environment, and parents
respond differently to children based on the child’s inherent biologic
characteristics.
\cite{Feldman3}

A child’s characteristics impact parenting, and parenting impacts children’s
development; these bidirectional cascades of influences continue over time
across development. Critically, parents’ behaviors in response to children are
driven by their interpretations and the meaning they make from the behavior.
For instance, a parent’s anxious handling may arise because of her perception
about the child’s birth complications; a parent may disengage from a child with
a difficult temperament because of the meaning she attaches to the child’s
fussy behavior.
\cite{Sameroff2009}

\subsubsection{Diathesis-Stress Model}
This model suggests that some individuals are more vulnerable to the impacts of
stress than others. Diatheses hereditary or constitutional predispositions
might include biologic, genetic, temperament-related, or cognitive factors that
predispose a child to being vulnerable to the influences of stress. Stresses
might include discrete life events (e.g., divorce), chronic stresses
(e.g., financial strain), or an accumulation of more minor daily stresses
(e.g., school assignments). In a developmentally supportive environment, this
model suggests that both resilient and vulnerable individuals are likely to do
well. In a challenging environment, resilient individuals would do well,
whereas vulnerable individuals would not. Those who have greater
predispositions to psychopathology may be overwhelmed by a small to moderate
environmental stress, whereas individuals with lower predispositions may
withstand higher lev- els of environmental stress without apparent effects on
their functioning.
\cite{Feldman3}

\subsubsection{Differential Susceptibility Theory}
This theory posits that individuals vary in their plasticity, or their level of
susceptibility to environmental influences. \cite{Belsky2021} Differential
susceptibility can contribute to positive and negative outcomes. Some children,
sometimes referred to as “orchids,” are very sensitive to their environment.
When they are in an environment that is highly supportive of their development
and well-being, they thrive; however, when they are in an environment that is
unsupportive of their development, they struggle. Other children, sometimes
referred to as “dandelions”, are less susceptible to environmental influences
and will do roughly the same regardless of how supportive their environment and
relationships are. Children can fall anywhere on the spectrum between these two
extremes. This degree of plasticity has been linked to differences in genetic
markers, including 5-HTTLPR, DRD4, and BDNF. \cite{Belsky2021} A child’s
susceptibility may differ depending on specific environmental influences and
the specific outcome being considered.
\cite{Feldman3}

\subsection{Caregiver-child relationships}
Primary caregivers impact a child’s biology through their interactions with the
child. The interactions a child has with, or supported by, the caregiver leave
a lasting mark on the child’s genome and brain structure. Through neural
pruning, a child’s neural connections are either reinforced or pruned based on
their experiences. Epigenetic effects, including those related to the
experience of early caretaking behavior, are also active during this
developmental period. \cite{Roth2011}

The influence of the childrearing environment dominates most current models of
development. Infants in hospitals and orphanages, devoid of opportunities for
attachment, have severe developmental deficits. Attachment refers to a
biologically determined tendency of a young child to seek proximity to the
parent during times of stress and to the relationship that allows securely
attached children to use their parents to reestablish a sense of wellbeing
after a stressful experience. Insecure attachment may be predictive of later
behavioral and learning problems. \cite{Nelson19}

At all stages of development, children progress optimally when they have adult
caregivers who pay attention to their verbal and nonverbal cues and respond
accordingly. In early infancy, such contingent responsiveness to signs of
overarousal or underarousal helps maintain infants in a state of quiet
alertness and fosters autonomic self-­ regulation. Consistent contingent
responses (reinforcement depending on the behavior of the other) to nonverbal
gestures create the groundwork for the shared attention and reciprocity that
are critical for later language and social development. \cite{Nelson19}

Caregivers scaffold infants’ cognitive, social, behavioral, emotional, and
physical development. Young children are not developmentally capable of
self-regulation, so caregivers play a crucial role in helping children cope
with negative affect by reading their cues, anticipating transitions,
redirecting their attention, and responding promptly to their needs. As
children gain experience with their caregivers coregulating their emotions,
they begin to internalize these regulation strategies and gradually develop the
ability to regulate independently. Sensitive and responsive parenting promotes
positive child outcomes in domains, including attachment, cognitive
development, social skills, and emotion regulation. A sensitive and responsive
caregiver is tuned into the child’s feelings and needs, and responds promptly
with actions that are in tune with the child’s feelings and needs throughout
everyday activities (e.g., feeding, play, bathing, changing diapers or
clothes). \cite{Feldman3}

\subsection{Child stress and trauma}
Even at young ages, many children are exposed to levels of stress and trauma
that can impact their development. Positive stress is considered a normal part
of healthy development. Heart rate, blood pressure, and stress hormones
temporarily increase. Tolerable stress involves greater stress response
activations, but stress elevations are still time limited with recovery once
the acute stressor passes; events such as disasters, the death of a loved one,
or divorce could be tolerable stressors. Caregiving relationships are key in
buffering the effect of these stressors, making stressors more manageable and
biologic stress responses subside. Toxic stress involves strong, frequent, and
prolonged stress system elevations that can cause lasting changes in
neurobiologic systems, having a detrimental effect on later physical and mental
health. \cite{Feldman3}

Children who have experienced trauma commonly have aggressive behavior,
irritability, and emotional withdrawal. Many children will reenact the trauma
they have experienced or witnessed either in vivo (e.g., toward their caregiver
or peers) or through play. Often these stresses happen in the context of 
caregiving relationships (e.g., child abuse or neglect, caregiver mental
illness or substance use disorder), which both magnifies the felt experience of
stress and lessens the potential for buffering of the stress through
relationships. When stressors are more intense, prolonged, repeated, and
unaddressed, they are likely to become toxic. When they occur in the context of
supportive social-emotional relationships, early detection, and effective
intervention, it is likely that these stressors will be tolerable.
\cite{Feldman3}

Social determinants of health are key contributors to stressors and traumas
that might lead to chronic elevations in their biologic stress response systems
and impact children’s development. Families who experience structural racism,
discrimination, or economic oppression often experience chronic elevations in
their biologic stress response systems that may contribute to a pervasive sense
of lack of safety and security. Parents who are experiencing these stressors
may understandably have less psychological capacity to support their children,
as it is exponentially harder to help a child feel safe and secure when you as
a parent do not feel safe and secure.
\cite{Feldman3}

A child’s relationship with a caregiver serves a particularly critical role
when the child is experiencing stress. When children have a relationship with a
caregiver that is secure, supportive, and attuned, the caregiver’s ability to
coregulate them supports their ability to withstand stress.
\cite{Feldman3}

For children who don’t have a secure attachment relationship, the biologic
stress response (e.g., HPA axis activity) may be pronounced; stress hormones
spike to higher levels and remain high for a longer period of time. In the
shorter term, the biologic stress response is adaptive, helping individuals to
face challenges in the environment. Persistent activation of the stress
response system without adequate recovery, however, is linked to lasting
neurobiologic changes such as reduced neuroplasticity and neurogenesis. These
neurobiologic changes are linked to lasting deleterious effects on an
individual’s mental and physical health. Parenting is the most influential
environmental factor that shapes individual differences in stress neurobiology
due to the role parents can play in coregulating children and the stress that
children experience when relationships are disrupted or insufficient to meet
children’s needs. \cite{Feldman3}


