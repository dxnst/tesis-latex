\section{Retraso en el desarrollo}

Delay in development occurs when the child fails to attain developmental
milestones as compared to peers from the same population. It is caused by
impairment in any of the following distinct domains, such as gross and fine
motor, speech and language, cognitive and performance, social, psychological,
sexual, and activities of daily living. \cite{DevelopmentalDelay}

``Developmental delay'' is a general descriptor of a broad phenotype that must
then be specified by carefully determining one or more elements linked with the
area of disrupted development. \cite{DevelopmentalDelay}

The term developmental disability includes a diverse group of life-long
physical and mental impairments that negatively affect an individual’s ability
to function as well as their peers. These conditions begin during childhood and
interfere with mobility, acquisition of self-care ability, communication
skills, social skills, general learning ability, and independent living.
\cite{Simms2023}

Developmental disabilities may be isolated, as in a child with impaired vision,
or may be multiple, as in a child with delays in motor, cognitive, language,
and social functioning. There may be considerable overlap in specific disorder
in terms of the affected functions. \cite{Simms2023}

In young children, developmental delays may result from a wide range of causes,
including early environmental understimulation, chronic physical illness,
neuromuscular disorders, central nervous system abnormalities, and genetic
syndromes. \cite{Simms2023}

Neurodevelopmental dysfunction places a child at risk for developmental,
cognitive, emotional, behavioral, psychosocial, and adaptive challenges.
Preschool-age children with neurodevelopmental or executive dysfunction may
manifest delays in developmental domains such as language, motor, self-help, or
social-emotional development and self-regulation. For the school-age child, an
area of particular focus is academic skill development. It is at this age that
disorders of learning are often diagnosed. \cite{Nelson49}

There are no prevalence estimates specifically for neurodevelopmental
dysfunction, but overall estimates for learning disorders range from 5\% to
10\% or more with a similar range reported for ADHD. These disorders frequently
co-occur.
