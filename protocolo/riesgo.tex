\section{Retraso en el desarrollo}

Delay in development occurs when the child fails to attain developmental
milestones as compared to peers from the same population. It is caused by
impairment in any of the following distinct domains, such as gross and fine
motor, speech and language, cognitive and performance, social, psychological,
sexual, and activities of daily living. \cite{DevelopmentalDelay}

"Developmental delay" is a general descriptor of a broad phenotype that must
then be specified by carefully determining one or more elements linked with the
area of disrupted development. \cite{DevelopmentalDelay}

The term developmental disability includes a diverse group of life-long
physical and mental impairments that negatively affect an individual’s ability
to function as well as their peers. These conditions begin during childhood and
interfere with mobility, acquisition of self-care ability, communication
skills, social skills, general learning ability, and independent living.
\cite{Simms2023}

Developmental disabilities may be isolated, as in a child with impaired vision,
or may be multiple, as in a child with delays in motor, cognitive, language,
and social functioning. There may be considerable overlap in specific disorder
in terms of the affected functions. \cite{Simms2023}

In young children, developmental delays may result from a wide range of causes,
including early environmental understimulation, chronic physical illness,
neuromuscular disorders, central nervous system abnormalities, and genetic
syndromes. \cite{Simms2023}

Neurodevelopmental dysfunction places a child at risk for developmental,
cognitive, emotional, behavioral, psychosocial, and adaptive challenges.
Preschool-age children with neurodevelopmental or executive dysfunction may
manifest delays in developmental domains such as language, motor, self-help, or
social-emotional development and self-regulation. For the school-age child, an
area of particular focus is academic skill development. It is at this age that
disorders of learning are often diagnosed. \cite{Nelson49}

There are no prevalence estimates specifically for neurodevelopmental
dysfunction, but overall estimates for learning disorders range from 5\% to
10\% or more with a similar range reported for ADHD. These disorders frequently
co-occur.

\section{Contexto guatemalteco en el desarrollo infantil}
Guatemala se caracteriza por ser un país multiétnico, pluricultural y
multilingüe. La población está constituida por los pueblos Maya, Garífuna,
Xinka y Mestizo. La mayor parte de la población guatemalteca habita en el área
urbana (54\%), aunque la mayoría de la población indígena reside en el área
rural (57\%). \cite{PoliticaInfanciaGuate}

De acuerdo al censo poblacional de 2018, en Guatemala habitan 2 millones 300
mil niñas y niños menores de 6 años, de los cuales, cerca de un millón vive en
condiciones de pobreza y 800 mil en extrema pobreza. \cite{INE}. Se estima que
los departamentos con mayor porcentaje de población con infancia temprana son
Totonicapán, Huehuetenango y El Quiché con respectivos porcentajes de 41.7\%,
41.4\% y 41.1\%. \cite{UNICEFAtlas}.

Según el mismo censo, el 3.8\% de las niñas, niños y adolescentes entre 4 a 17
años tenían algún tipo de dificultad visual, auditiva, física, de
concentración, de auto cuidado o de comunicación. \cite{INE}

En Guatemala, la niñez durante la primera infancia no logra alcanzar su
potencial de desarrollo integral en todas sus dimensiones, es decir, el
desarrollo sensorial, cognitivo, físico, motor, del lenguaje, socioemocional y
espiritual que le permita llevar una vida sana, productiva y digna.
\cite{PoliticaInfanciaGuate}

Guatemala is the largest economy in Central America in terms of population
(estimated at 17.6 million in 2023) and economic activity (with a GDP of 104.4
US billion dollars in 2023). However, this economic growth has not translated
into a significant reduction in poverty. In 2023, it was estimated that 55\% of
the population lived in poverty, and the informal economy represented 49\% of
GDP, with 71.1\% of the employed population working in the informal sector.
\cite{WorldBankGuate} La pobreza y las experiencias adversas de la infancia
tienen efectos fisiológicos y epigenéticos a largo plazo en el desarrollo
cerebral y la cognición. \cite{Luby_2015} \cite{Noble_2015}

According to data from 2021-2022, Guatemala has scored 50\% in the Early
Childhood Development Index. This means that just half of children aged 24-59
months are developmentally on track in health, learning, and psychosocial
well-being. Rates of adequate development differ by population: whereas 57.9\%
of non-Indigenous children show adequate development, only 45\% of Indigenous
children do. \cite{SESAN2022} Cuando se observa el índice en función de la
riqueza del hogar se encuentra una brecha aún mayor de 23 puntos porcentuales
entre los hogares con mayor y menor riqueza. \cite{UNICEFGuate}

En cuanto al acceso que tienen las niñas y los niños de primera infancia a
oportunidades de aprendizaje, en Guatemala para el 2020, la tasa neta de
cobertura en el nivel inicial fue de 1.1\% a nivel nacional. En el mismo año,
habían 597,195 niñas y niños inscritos en preprimaria con una cobertura de
60.8\% a nivel nacional, los cuales representan un 13.8\% del total de niñas,
niños y adolescentes matriculados en los distintos niveles educativos.
\cite{PoliticaInfanciaGuate} La cobertura en el nivel de educación inicial para
niñas y niños menores de 4 años es baja, y la del nivel de educación
preprimaria para la atención de niñas y niños de 4 a 6 años es de 64.4\%.
\cite{MineducEstadistica}

La cobertura de educación en niñez con discapacidad representa un desafío para
el país. Según la ENDIS 2016, el 76\% de las niñas y niños entre 5 y 18 años
con algún tipo de discapacidad asisten a la escuela, siendo dicho porcentaje
mayor en el área urbana (90\%) que en el área rural (61\%). La niñez con
limitaciones significativas en el funcionamiento físico o cognitivo tienen
menor probabilidad de ser inscritos en la escuela. \cite{PoliticaInfanciaGuate}

Al analizar la relación entre el desarrollo infantil temprano y el estado
nutricional de la primera infancia se obtuvieron resultados acordes con
estudios previos que muestran que, en general, los niños y las niñas con
desnutrición crónica tienden a mostrar menor desarrollo. En la encuesta de
línea base de la Cruzada Nacional por la Nutrición 2021/2022 \cite{SESAN2022}
esta relación es, efectivamente, mayor para los niños y niñas que no presentan
desnutrición crónica (55.6\%), en contraste con quienes sí padecen
este tipo de malnutrición (43.3\%).

Se estima que 46.5\% de los niños guatemaltecos menores de 5 años viven en
desnutrición crónica. La desnutrición crónica refleja desigualdades
importantes, en la medida que el porcentaje de desnutridos crónicos es mayor en
áreas rurales (53\% comparado con 34.6\% en las urbanas), en población indígena
(58\% comparado con 34.2\% en la no indígena), en hogares en que las madres no
tienen escolaridad (67.0\% y 19.1\% en niños cuya madre tiene educación
superior), en hogares con menor riqueza económica (65.9\% en los de quintil
inferior y 17.4\% en hogar en el quintil superior de riqueza) y un menor
espaciamiento entre embarazos (57.0\% comparado a 39.6\% mayor espaciamiento)
\cite{EnMaternoInfantil}

En el país, solo el 63.1\% de niñas y niños reciben lactancia materna dentro de
la primera hora de nacidos, el 53.2\% reciben lactancia materna exclusiva entre
los 0 a 6 meses de edad y la duración promedio de la lactancia materna
exclusiva es de 2.8 meses. El 55.7\% de las niñas y niños entre los 6 a 23
meses que son amamantados reciben cuatro o más grupos de alimentos y el 71.2\%
de la niñez no amamantada, en este mismo rango de edad, reciben una frecuencia
mínima de comida. \cite{EnMaternoInfantil} \cite{PoliticaInfanciaGuate}

Pese a que el vínculo materno-infantil o entre el cuidador y el niño es
esencial para un desarrollo infantil positivo, muchos niños guatemaltecos
experimentan violencia proveniente de sus cuidadores. En un estudio elaborado a
finales del 2019 en 52 comunidades de los departamentos de Sololá,
Alta Verapaz, Baja Verapaz, Chimaltenango, Quetzaltenango y San Marcos, 8 de
cada 10 adultos opinan que en su comunidad la forma más usada para corregir a
las hijas e hijos es a través del castigo como cincho, chicote, vara, golpes o
gritos. La mitad de los adultos piensa que, si los padres no corrigen o
disciplinan de esta manera, es por falta de carácter.
\cite{PoliticaInfanciaGuate} \cite{UNICEFComportamientosNinez}
