\chapter*{Abstract}
\pagenumbering{roman}
Neurodevelopmental disorders in early childhood represent a global health 
crisis, with approximately 250 million children under 5 years of age at risk 
of not reaching their developmental potential worldwide. In Guatemala, the 
situation is concerning, with only 49.8\% of children aged 24-59 months 
showing adequate development in health, learning, and psychosocial well-being, 
and significant disparities between indigenous and non-indigenous populations. 
However, comprehensive evidence on specific risk factors associated with 
neurodevelopmental outcomes in Guatemala's diverse regional contexts remains 
limited, particularly in Quetzaltenango. Here we show that sociodemographic, 
economic, and environmental factors are significantly associated with 
neurodevelopmental risk patterns in a sample of 1,725 children under 5 years 
of age attending primary health care services in the Quetzaltenango district. 
Using the Ages and Stages Questionnaire-3 (ASQ-3), we found that 66.96\% of 
children had adequate global development, while 33.04\% showed risk in any 
domain and 5.22\% high risk, with gross motor skills showing the greatest 
vulnerability (16.7\% at risk). Indigenous ethnicity was associated with 
increased risk in problem-solving (OR=1.50, 95\% CI: 1.05-2.16) and 
socio-individual development (OR=1.53, 95\% CI: 1.06-2.21), while rural 
residence showed robust associations with risk across four domains, including 
communication (OR=2.09) and problem-solving (OR=2.77). Maternal education 
level was significantly associated with all developmental domains ($p<0.001$), 
chronic growth retardation affected gross motor and socio-individual 
development, and increased screen time exposure was associated with 
developmental risk across all domains. These findings provide evidence-based 
insights for designing targeted interventions that address socioeconomic, 
educational and nutritional disparities to improve early childhood 
development in Guatemala's most vulnerable populations.

\textbf{Keywords:} Child Development, Neuropsychological Tests, Developmental
Disabilities, Mass Screening, Risk Assessment.

\chapter*{Resumen}

Los trastornos del neurodesarrollo en la primera infancia representan una 
crisis de salud global, con aproximadamente 250 millones de niños menores de 5 
años en riesgo de no alcanzar su potencial de desarrollo a nivel mundial. En 
Guatemala, la situación es preocupante, con solo el 49.8\% de los 
niños de 24-59 meses mostrando un desarrollo adecuado en salud, aprendizaje y 
bienestar psicosocial, y disparidades significativas entre poblaciones 
indígenas y no indígenas. Sin embargo, la evidencia integral sobre los factores 
de riesgo específicos asociados con los resultados del neurodesarrollo en los 
diversos contextos regionales de Guatemala sigue siendo limitada, 
particularmente en Quetzaltenango. Aquí mostramos que los 
factores sociodemográficos, económicos y ambientales están significativamente 
asociados con patrones de riesgo del neurodesarrollo en una muestra de 1,725 
niños menores de 5 años que asisten a servicios de atención primaria en el 
distrito de Quetzaltenango. Utilizando el Cuestionario de Edades y Etapas-3 
(ASQ-3), encontramos que el 66.96\% de los niños tenían un desarrollo global 
adecuado, mientras que el 33.04\% mostraron riesgo en cualquier dominio y el 
5.22\% alto riesgo, con las habilidades motoras gruesas mostrando la mayor 
vulnerabilidad (16.7\% en riesgo). La etnicidad indígena se asoció con mayor 
riesgo en resolución de problemas (OR=1.50, IC 95\%: 1.05-2.16) y desarrollo 
socio-individual (OR=1.53, IC 95\%: 1.06-2.21), mientras que la residencia 
rural mostró asociaciones robustas con riesgo en cuatro dominios, incluyendo 
comunicación (OR=2.09) y resolución de problemas (OR=2.77). El nivel educativo 
materno se asoció significativamente con todos los dominios del desarrollo 
($p<0.001$), el retardo de crecimiento crónico afectó el desarrollo motor 
grueso y socio-individual, y el aumento del tiempo de exposición a pantallas 
se asoció con riesgo de desarrollo en todos los dominios. Estos hallazgos 
proporcionan conocimientos basados en evidencia para diseñar intervenciones 
dirigidas que aborden las disparidades socioeconómicas, educativas y
nutricionales  para mejorar el desarrollo infantil temprano en las poblaciones
más  vulnerables de Guatemala.

\textbf{Palabras clave:} Desarrollo Infantil, Pruebas Neuropsicológicas, 
Discapacidades del Desarrollo, Tamizaje Masivo, Evaluación de Riesgos.
